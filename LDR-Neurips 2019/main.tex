\documentclass{article}

% if you need to pass options to natbib, use, e.g.:
%     \PassOptionsToPackage{numbers, compress}{natbib}
% before loading neurips_2019

% ready for submission
% \usepackage{neurips_2019}

% to compile a preprint version, e.g., for submission to arXiv, add add the
% [preprint] option:
%     \usepackage[preprint]{neurips_2019}

% to compile a camera-ready version, add the [final] option, e.g.:
     \usepackage{neurips_2019}
     \usepackage{xr}
    \externaldocument{supplementary}

    %\RequirePackage[l2tabu,orthodox]{nag}
     
     % SETUP
     \usepackage[utf8]{inputenc}
     \usepackage{algorithmic}
     \usepackage{algorithm}
     
     % {{{ etex }}}
     \usepackage{etex}
     \usepackage{times}
     % {{{ common }}}
     \usepackage{xspace,enumerate}
     \usepackage[dvipsnames]{xcolor}
     %\usepackage[T1]{fontenc}
     %\usepackage[full]{textcomp}
     % {{{ babelamerican }}}
     \usepackage[american]{babel}
     % {{{ mathtools }}}99
     \usepackage{mathtools}
     \usepackage{thmtools}
     \usepackage{thm-restate}
     % {{{ boldmath }}}
     % fix for "too many math alphabets" problem
     \newcommand\hmmax{0} % default 3

     % \usepackage{bm}
     % \usepackage{stmaryrd}
     % {{{ amsthm }}}
     \usepackage{amsthm}
     \newtheorem{theorem}{Theorem}[section]
     \newtheorem*{theorem*}{Theorem}
     \newtheorem{subclaim}{Claim}[theorem]
     \newtheorem{proposition}[theorem]{Proposition}
     \newtheorem*{proposition*}{Proposition}
     \newtheorem{lemma}[theorem]{Lemma}
     \newtheorem*{lemma*}{Lemma}
     \newtheorem{corollary}[theorem]{Corollary}
     \newtheorem*{conjecture*}{Conjecture}
     \newtheorem{fact}[theorem]{Fact}
     \newtheorem*{fact*}{Fact}
     \newtheorem{hypothesis}[theorem]{Hypothesis}
     \newtheorem*{hypothesis*}{Hypothesis}
     \newtheorem{conjecture}[theorem]{Conjecture}
     \theoremstyle{definition}
     \newtheorem{definition}[theorem]{Definition}
     \newtheorem*{definition*}{Definition}
     \newtheorem{algorithms}{Algorithm}
     \newtheorem{construction}[theorem]{Construction}
     \newtheorem{example}[theorem]{Example}
     \newtheorem{question}[theorem]{Question}
     \newtheorem{openquestion}[theorem]{Open Question}
     \newtheorem{problem}[theorem]{Problem}
     \newtheorem{model}[theorem]{Model}
     \newtheorem{protocol}[theorem]{Protocol}
     \newtheorem{assumption}[theorem]{Assumption}
     \theoremstyle{remark}
     \newtheorem{claim}[theorem]{Claim}
     \newtheorem*{claim*}{Claim}
     \newtheorem{remark}[theorem]{Remark}
     \newtheorem*{remark*}{Remark}
     \newtheorem{observation}[theorem]{Observation}
     \newtheorem*{observation*}{Observation}
     % {{{ geometry-nice }}}
%     \usepackage[
%     letterpaper,
%     top=1.2in,
%     bottom=1.2in,
%     left=1in,
%     right=1in]{geometry}
     % {{{ newpx }}}
     %\usepackage{newpxtext} % T1, lining figures in math, osf in text
     %\usepackage{textcomp} % required for special glyphs
     %\usepackage[varg,bigdelims]{newpxmath}
     %\usepackage[scr=rsfso]{mathalfa}% \mathscr is fancier than \mathcal
     \usepackage{bm} % load after all math to give access to bold math
     % \useosf %no longer needed
     %\linespread{1.1}% Give Palatino more leading (space between lines)
     %\let\mathbb\varmathbb
     % {{{ microtype }}}
     \usepackage{microtype}
     \usepackage[
     pagebackref,
     % letterpaper=true,
     colorlinks=true,
     urlcolor=blue,
     linkcolor=blue,
     citecolor=OliveGreen,
     ]{hyperref}
     \usepackage[capitalise,nameinlink]{cleveref}
     \crefname{lemma}{Lemma}{Lemmas}
     \crefname{fact}{Fact}{Facts}
     \crefname{theorem}{Theorem}{Theorems}
     \crefname{corollary}{Corollary}{Corollaries}
     \crefname{claim}{Claim}{Claims}
     \crefname{example}{Example}{Examples}
     \crefname{algorithms}{Algorithm}{Algorithms}
     \crefname{problem}{Problem}{Problems}
     \crefname{definition}{Definition}{Definitions}
     \usepackage{paralist}
     \usepackage{turnstile}
     \usepackage{mdframed}
     \usepackage{tikz}
     \usepackage{caption}
     \DeclareCaptionType{Algorithm}
     \usepackage{newfloat}
     
     % MACROS
     
     %\newcommand{\Authornote}[2]{}
     %\newcommand{\Authornotecolored}[3]{{\leavevmode\color{#1} #2: #3}}
     \newcommand{\Authornotecolored}[3]{}
     \newcommand{\fixme}[1]{\Authornotecolored{red}{[FIX ME}{\leavevmode\color{blue}#1]}}
     %\newcommand{\fixme}[1]{}
     % \newcommand{\Authorcomment}[2]{}
     % \newcommand{\Authorfnote}[2]{}
     % \newcommand{\Dnote}{\Authornote{D}}
     % \newcommand{\Dcomment}{\Authorcomment{D}}
     % \newcommand{\Dfnote}{\Authorfnote{D}}
     % \newcommand{\Pnote}[1]{\Authornotecolored{blue}{[P}{#1]}}
     \newcommand{\SK}[1]{}
     \newcommand{\Pnote}[1]{}
     % \definecolor{forestgreen(traditional)}{rgb}{0.0, 0.27, 0.13}
     % \newcommand{\SK}[1]{{\color{ForestGreen}(S: #1)}}
     % \newcommand{\inote}{\Inote}
     % \newcommand{\Gnote}{\Authornote{G}}
     % \newcommand{\Gfnote}{\Authorfnote{G}}
     % \newcommand{\Pfnote}{\Authorfnote{P}}
     % \newcommand{\PRnote}{\Authornote{PR}}
     % \newcommand{\PRfnote}{\Authorfnote{PR}}
     % \newcommand{\PTnote}{\Authornote{PT}}
     % \newcommand{\PTfnote}{\Authorfnote{PT}}
     % \newcommand{\PGnote}{\Authornote{PG}}
     % \newcommand{\PGfnote}{\Authorfnote{PG}}
     
     % \newcommand{\Bnote}{\Authornote{B}}
     % \newcommand{\Bfnote}{\Authorfnote{B}}
     % \newcommand{\Jnote}{\Authornotecolored{forestgreen(traditional)}{J}}
     % \newcommand{\Jfnote}{\Authorfnote{J}}
     % \newcommand{\Jcomment}{\Authorcomment{J}}
     % \newcommand{\Mnote}{\Authornote{M}}
     % \newcommand{\Mfnote}{\Authorfnote{M}}
     % \newcommand{\Rnote}{\Authornote{M}}
     % \newcommand{\Rfnote}{\Authorfnote{M}}
     % \newcommand{\Nnote}{\Authornote{N}}
     % \newcommand{\Nfnote}{\Authorfnote{N}}
     % \newcommand{\Snote}{\Authornote{S}}
     % \newcommand{\Sfnote}{\Authorfnote{S}}
     \usepackage{boxedminipage}
     % example:
     % \center \noindent\begin{boxedminipage}{1.0\linewidth}}
     % content
     % \end{boxedminipage}
     % \noindent
     % {{{ parentheses }}}
     % various bracket-like commands
     % round parentheses
      \newcommand{\paren}[1]{(#1)}
      \newcommand{\Paren}[1]{\left(#1\right)}
     % \newcommand{\bigparen}[1]{\big(#1\big)}
     % \newcommand{\Bigparen}[1]{\Big(#1\Big)}
     % % square brackets
     % \newcommand{\brac}[1]{[#1]}
     % \newcommand{\Brac}[1]{\left[#1\right]}
     % \newcommand{\bigbrac}[1]{\big[#1\big]}
     % \newcommand{\Bigbrac}[1]{\Big[#1\Big]}
     % \newcommand{\Biggbrac}[1]{\Bigg[#1\Bigg]}
     % % absolute value
     % \newcommand{\abs}[1]{\lvert#1\rvert}
     % \newcommand{\Abs}[1]{\left\lvert#1\right\rvert}
     % \newcommand{\bigabs}[1]{\big\lvert#1\big\rvert}
     % \newcommand{\Bigabs}[1]{\Big\lvert#1\Big\rvert}
     % % cardinality
     % \newcommand{\card}[1]{\lvert#1\rvert}
     % \newcommand{\Card}[1]{\left\lvert#1\right\rvert}
     % \newcommand{\bigcard}[1]{\big\lvert#1\big\rvert}
     % \newcommand{\Bigcard}[1]{\Big\lvert#1\Big\rvert}
     % % set
      \newcommand{\set}[1]{\{#1\}}
     \newcommand{\Set}[1]{\left\{#1\right\}}
     % \newcommand{\bigset}[1]{\big\{#1\big\}}
     % \newcommand{\Bigset}[1]{\Big\{#1\Big\}}
     % % norm
      \newcommand{\norm}[1]{\lVert#1\rVert}
      \newcommand{\Norm}[1]{\left\lVert#1\right\rVert}
     % \newcommand{\bignorm}[1]{\big\lVert#1\big\rVert}
     % \newcommand{\Bignorm}[1]{\Big\lVert#1\Big\rVert}
     % \newcommand{\Biggnorm}[1]{\Bigg\lVert#1\Bigg\rVert}
     % % 2-norm
     % \newcommand{\normt}[1]{\norm{#1}_2}
     % \newcommand{\Normt}[1]{\Norm{#1}_2}
     % \newcommand{\bignormt}[1]{\bignorm{#1}_2}
     % \newcommand{\Bignormt}[1]{\Bignorm{#1}_2}
     % % 2-norm squared
     % \newcommand{\snormt}[1]{\norm{#1}^2_2}
     % \newcommand{\Snormt}[1]{\Norm{#1}^2_2}
     % \newcommand{\bigsnormt}[1]{\bignorm{#1}^2_2}
     % \newcommand{\Bigsnormt}[1]{\Bignorm{#1}^2_2}
     % % norm squared
     % \newcommand{\snorm}[1]{\norm{#1}^2}
     % \newcommand{\Snorm}[1]{\Norm{#1}^2}
     % \newcommand{\bigsnorm}[1]{\bignorm{#1}^2}
     % \newcommand{\Bigsnorm}[1]{\Bignorm{#1}^2}
     % % 1-norm
     % \newcommand{\normo}[1]{\norm{#1}_1}
     % \newcommand{\Normo}[1]{\Norm{#1}_1}
     % \newcommand{\bignormo}[1]{\bignorm{#1}_1}
     % \newcommand{\Bignormo}[1]{\Bignorm{#1}_1}
     % % infty-norm
      \newcommand{\normi}[1]{\norm{#1}_\infty}
     % \newcommand{\Normi}[1]{\Norm{#1}_\infty}
     % \newcommand{\bignormi}[1]{\bignorm{#1}_\infty}
     % \newcommand{\Bignormi}[1]{\Bignorm{#1}_\infty}
     % % inner product
      \newcommand{\iprod}[1]{\langle#1\rangle}
     % \newcommand{\Iprod}[1]{\left\langle#1\right\rangle}
     % \newcommand{\bigiprod}[1]{\big\langle#1\big\rangle}
     % \newcommand{\Bigiprod}[1]{\Big\langle#1\Big\rangle}
     % % {{{ probability }}}
     % % expectation, probability, variance
     % \newcommand{\Esymb}{\mathbb{E}}
     % \newcommand{\Psymb}{\mathbb{P}}
     % \newcommand{\Vsymb}{\mathbb{V}}
     % \DeclareMathOperator*{\E}{\Esymb}
     % \DeclareMathOperator*{\Var}{\Vsymb}
     % \DeclareMathOperator*{\ProbOp}{\Psymb}
     % \renewcommand{\Pr}{\ProbOp}
     % %\newcommand{\given}{\;\middle\vert\;}
     % \newcommand{\given}{\mathrel{}\middle\vert\mathrel{}}
     % %\newcommand{\given}{\mathrel{}\middle|\mathrel{}}
     % % {{{ miscmacros }}}
     % % middle delimiter in the definition of a set
     % \newcommand{\suchthat}{\;\middle\vert\;}
     % % tensor product
     % \newcommand{\tensor}{\otimes}
     % % add explanations to math displays
     % \newcommand{\where}{\text{where}}
     % \newcommand{\textparen}[1]{\text{(#1)}}
     % \newcommand{\using}[1]{\textparen{using #1}}
     % \newcommand{\smallusing}[1]{\text{(\small using #1)}}
     % \newcommand{\by}[1]{\textparen{by #1}}
     % % spectral order (Loewner order)
     % \newcommand{\sge}{\succeq}
     % \newcommand{\sle}{\preceq}
     % % smallest and largest eigenvalue
     % \newcommand{\lmin}{\lambda_{\min}}
     % \newcommand{\lmax}{\lambda_{\max}}
     % \newcommand{\signs}{\set{1,-1}}
     % \newcommand{\varsigns}{\set{\pm 1}}
     % \newcommand{\maximize}{\mathop{\textrm{maximize}}}
     % \newcommand{\minimize}{\mathop{\textrm{minimize}}}
     % \newcommand{\subjectto}{\mathop{\textrm{subject to}}}
     % \renewcommand{\ij}{{ij}}
     % % symmetric difference
     % \newcommand{\symdiff}{\Delta}
     % \newcommand{\varsymdiff}{\bigtriangleup}
     % % set of bits
     % \newcommand{\bits}{\{0,1\}}
     % \newcommand{\sbits}{\{\pm1\}}
     % % no stupid bullets for itemize environmentx
     % % \renewcommand{\labelitemi}{--}
     % % control white space of list and display environments
     % \newcommand{\listoptions}{\labelsep0mm\topsep-0mm\itemindent-6mm\itemsep0mm}
     % \newcommand{\displayoptions}[1]{\abovedisplayshortskip#1mm\belowdisplayshortskip#1mm\abovedisplayskip#1mm\belowdisplayskip#1mm}
     % % short for emptyset
     % %\newcommand{\eset}{\emptyset}
     % % moved to mathabbreviations
     % % short for epsilon
     % %\newcommand{\e}{\epsilon}
     % % moved to mathabbreviations
     % % super index with parentheses
     % \newcommand{\super}[2]{#1^{\paren{#2}}}
     % \newcommand{\varsuper}[2]{#1^{\scriptscriptstyle\paren{#2}}}
     % % tensor power notation
     % \newcommand{\tensorpower}[2]{#1^{\tensor #2}}
     % % multiplicative inverse
     % \newcommand{\inv}[1]{{#1^{-1}}}
     % % dual element
     % \newcommand{\dual}[1]{{#1^*}}
     % % subset
     % %\newcommand{\sse}{\subseteq}
     % % moved to mathabbreviations
     % % vertical space in math formula
     % \newcommand{\vbig}{\vphantom{\bigoplus}}
     % % setminus
     % \newcommand{\sm}{\setminus}
     % % define something by an equation (display)
     % \newcommand{\defeq}{\stackrel{\mathrm{def}}=}
     % % define something by an equation (inline)
     % \newcommand{\seteq}{\mathrel{\mathop:}=}
     % % declare function f by $f \from X \to Y$
      \newcommand{\from}{\colon}
     % % big middle separator (for conditioning probability spaces)
     % \newcommand{\bigmid}{~\big|~}
     % \newcommand{\Bigmid}{~\Big|~}
     % \newcommand{\Mid}{\;\middle\vert\;}
     % % better vector definition and some variations
     % %\renewcommand{\vec}[1]{{\bm{#1}}}
     % \newcommand{\bvec}[1]{\bar{\vec{#1}}}
     % \newcommand{\pvec}[1]{\vec{#1}'}
     % \newcommand{\tvec}[1]{{\tilde{\vec{#1}}}}
     % % punctuation at the end of a displayed formula
      \newcommand{\mper}{\,.}
      \newcommand{\mcom}{\,,}
     % % inner product for matrices
     % \newcommand\bdot\bullet
     % % transpose
     % \newcommand{\trsp}[1]{{#1}^\dagger}
     % % indicator function / vector
     % \DeclareMathOperator{\Ind}{\mathbf 1}
     % % place a qed symbol inside display formula
     % %\qedhere
     % % {{{ mathoperators }}}
     % \DeclareMathOperator{\Inf}{Inf}
     % \DeclareMathOperator{\Tr}{Tr}
     % %\newcommand{\Tr}{\mathrm{Tr}}
     % \DeclareMathOperator{\SDP}{SDP}
     % \DeclareMathOperator{\sdp}{sdp}
     % \DeclareMathOperator{\val}{val}
     % \DeclareMathOperator{\LP}{LP}
     % \DeclareMathOperator{\OPT}{OPT}
     % \DeclareMathOperator{\opt}{opt}
     % \DeclareMathOperator{\vol}{vol}
     % \DeclareMathOperator{\poly}{poly}
     % \DeclareMathOperator{\qpoly}{qpoly}
     % \DeclareMathOperator{\qpolylog}{qpolylog}
     % \DeclareMathOperator{\qqpoly}{qqpoly}
     % \DeclareMathOperator{\argmax}{argmax}
     % \DeclareMathOperator{\polylog}{polylog}
     % \DeclareMathOperator{\supp}{supp}
     % \DeclareMathOperator{\dist}{dist}
     % \DeclareMathOperator{\sign}{sign}
     % \DeclareMathOperator{\conv}{conv}
     % \DeclareMathOperator{\Conv}{Conv}
     % \DeclareMathOperator{\rank}{rank}
     % % operators with limits
     % \DeclareMathOperator*{\median}{median}
     % \DeclareMathOperator*{\Median}{Median}
     % % smaller summation/product symbols
     % \DeclareMathOperator*{\varsum}{{\textstyle \sum}}
     % \DeclareMathOperator{\tsum}{{\textstyle \sum}}
     % \let\varprod\undefined
     % \DeclareMathOperator*{\varprod}{{\textstyle \prod}}
     % \DeclareMathOperator{\tprod}{{\textstyle \prod}}
     % % {{{ textabbreviations }}}
     % % some abbreviations
     % \newcommand{\ie}{i.e.,\xspace}
     % \newcommand{\eg}{e.g.,\xspace}
     % \newcommand{\Eg}{E.g.,\xspace}
     % \newcommand{\phd}{Ph.\,D.\xspace}
     % \newcommand{\msc}{M.\,S.\xspace}
     % \newcommand{\bsc}{B.\,S.\xspace}
     % \newcommand{\etal}{et al.\xspace}
     % \newcommand{\iid}{i.i.d.\xspace}
     % % {{{ foreignwords }}}
     % \newcommand\naive{na\"{\i}ve\xspace}
     % \newcommand\Naive{Na\"{\i}ve\xspace}
     % \newcommand\naively{na\"{\i}vely\xspace}
     % \newcommand\Naively{Na\"{\i}vely\xspace}
     % % {{{ names }}}
     % % Hungarian/Polish/East European names
     % \newcommand{\Erdos}{Erd\H{o}s\xspace}
     % \newcommand{\Renyi}{R\'enyi\xspace}
     % \newcommand{\Lovasz}{Lov\'asz\xspace}
     % \newcommand{\Juhasz}{Juh\'asz\xspace}
     % \newcommand{\Bollobas}{Bollob\'as\xspace}
     % \newcommand{\Furedi}{F\"uredi\xspace}
     % \newcommand{\Komlos}{Koml\'os\xspace}
     % \newcommand{\Luczak}{\L uczak\xspace}
     % \newcommand{\Kucera}{Ku\v{c}era\xspace}
     % \newcommand{\Szemeredi}{Szemer\'edi\xspace}
     % \newcommand{\Hastad}{H{\aa}stad\xspace}
     % \newcommand{\Hoelder}{H\"{o}lder\xspace}
     % \newcommand{\Holder}{\Hoelder}
     % \newcommand{\Brandao}{Brand\~ao\xspace}
     % % {{{ numbersets }}}
     % % number sets
     % \newcommand{\Z}{\mathbb Z}
     % \newcommand{\N}{\mathbb N}
      \newcommand{\R}{\mathbb R}
     % \newcommand{\C}{\mathbb C}
     % \newcommand{\Rnn}{\R_+}
     % \newcommand{\varR}{\Re}
     % \newcommand{\varRnn}{\varR_+}
     % \newcommand{\varvarRnn}{\R_{\ge 0}}
     % % {{{ problems }}}
     % % macros to denote computational problems
     % % use texorpdfstring to avoid problems with hyperref (can use problem
     % % macros also in headings
     % \newcommand{\problemmacro}[1]{\texorpdfstring{\textup{\textsc{#1}}}{#1}\xspace}
     % \newcommand{\pnum}[1]{{\footnotesize #1}}
     % % list of problems
     % \newcommand{\uniquegames}{\problemmacro{unique games}}
     % \newcommand{\maxcut}{\problemmacro{max cut}}
     % \newcommand{\multicut}{\problemmacro{multi cut}}
     % \newcommand{\vertexcover}{\problemmacro{vertex cover}}
     % \newcommand{\balancedseparator}{\problemmacro{balanced separator}}
     % \newcommand{\maxtwosat}{\problemmacro{max \pnum{3}-sat}}
     % \newcommand{\maxthreesat}{\problemmacro{max \pnum{3}-sat}}
     % \newcommand{\maxthreelin}{\problemmacro{max \pnum{3}-lin}}
     % \newcommand{\threesat}{\problemmacro{\pnum{3}-sat}}
     % \newcommand{\labelcover}{\problemmacro{label cover}}
     % \newcommand{\setcover}{\problemmacro{set cover}}
     % \newcommand{\maxksat}{\problemmacro{max $k$-sat}}
     % \newcommand{\mas}{\problemmacro{maximum acyclic subgraph}}
     % \newcommand{\kwaycut}{\problemmacro{$k$-way cut}}
     % \newcommand{\sparsestcut}{\problemmacro{sparsest cut}}
     % \newcommand{\betweenness}{\problemmacro{betweenness}}
     % \newcommand{\uniformsparsestcut}{\problemmacro{uniform sparsest cut}}
     % \newcommand{\grothendieckproblem}{\problemmacro{Grothendieck problem}}
     % \newcommand{\maxfoursat}{\problemmacro{max \pnum{4}-sat}}
     % \newcommand{\maxkcsp}{\problemmacro{max $k$-csp}}
     % \newcommand{\maxdicut}{\problemmacro{max dicut}}
     % \newcommand{\maxcutgain}{\problemmacro{max cut gain}}
     % \newcommand{\smallsetexpansion}{\problemmacro{small-set expansion}}
     % \newcommand{\minbisection}{\problemmacro{min bisection}}
     % \newcommand{\minimumlineararrangement}{\problemmacro{minimum linear arrangement}}
     % \newcommand{\maxtwolin}{\problemmacro{max \pnum{2}-lin}}
     % \newcommand{\gammamaxlin}{\problemmacro{$\Gamma$-max \pnum{2}-lin}}
     % \newcommand{\basicsdp}{\problemmacro{basic sdp}}
     % \newcommand{\dgames}{\problemmacro{$d$-to-1 games}}
     % \newcommand{\maxclique}{\problemmacro{max clique}}
     % \newcommand{\densestksubgraph}{\problemmacro{densest $k$-subgraph}}
     % % {{{ alphabet }}}
     \newcommand{\E}{\mathbb{E}}
     \newcommand{\N}{\mathbb{N}}
      \newcommand{\cA}{\mathcal A}
      \newcommand{\cB}{\mathcal B}
      \newcommand{\cC}{\mathcal C}
     % \newcommand{\cD}{\mathcal D}
     % \newcommand{\cE}{\mathcal E}
     % \newcommand{\cF}{\mathcal F}
     % \newcommand{\cG}{\mathcal G}
     % \newcommand{\cH}{\mathcal H}
      \newcommand{\cI}{\mathcal I}
     % \newcommand{\cJ}{\mathcal J}
     % \newcommand{\cK}{\mathcal K}
     % \newcommand{\cL}{\mathcal L}
     % \newcommand{\cM}{\mathcal M}
      \newcommand{\cN}{\mathcal N}
      \newcommand{\cO}{\mathcal O}
     % \newcommand{\cP}{\mathcal P}
     % \newcommand{\cQ}{\mathcal Q}
     % \newcommand{\cR}{\mathcal R}
      \newcommand{\cS}{\mathcal S}
     % \newcommand{\cT}{\mathcal T}
     % \newcommand{\cU}{\mathcal U}
     % \newcommand{\cV}{\mathcal V}
     % \newcommand{\cW}{\mathcal W}
     % \newcommand{\cX}{\mathcal X}
     % \newcommand{\cY}{\mathcal Y}
     % \newcommand{\cZ}{\mathcal Z}
     % \newcommand{\scrA}{\mathscr A}
     % \newcommand{\scrB}{\mathscr B}
     % \newcommand{\scrC}{\mathscr C}
     % \newcommand{\scrD}{\mathscr D}
     % \newcommand{\scrE}{\mathscr E}
     % \newcommand{\scrF}{\mathscr F}
     % \newcommand{\scrG}{\mathscr G}
     % \newcommand{\scrH}{\mathscr H}
     % \newcommand{\scrI}{\mathscr I}
     % \newcommand{\scrJ}{\mathscr J}
     % \newcommand{\scrK}{\mathscr K}
     % \newcommand{\scrL}{\mathscr L}
     % \newcommand{\scrM}{\mathscr M}
     % \newcommand{\scrN}{\mathscr N}
     % \newcommand{\scrO}{\mathscr O}
     % \newcommand{\scrP}{\mathscr P}
     % \newcommand{\scrQ}{\mathscr Q}
     % \newcommand{\scrR}{\mathscr R}
     % \newcommand{\scrS}{\mathscr S}
     % \newcommand{\scrT}{\mathscr T}
     % \newcommand{\scrU}{\mathscr U}
     % \newcommand{\scrV}{\mathscr V}
     % \newcommand{\scrW}{\mathscr W}
     % \newcommand{\scrX}{\mathscr X}
     % \newcommand{\scrY}{\mathscr Y}
     % \newcommand{\scrZ}{\mathscr Z}
     % \newcommand{\bbB}{\mathbb B}
     % \newcommand{\bbS}{\mathbb S}
     % \newcommand{\bbR}{\mathbb R}
     % \newcommand{\bbZ}{\mathbb Z}
     % \newcommand{\bbI}{\mathbb I}
     % \newcommand{\bbQ}{\mathbb Q}
     % \newcommand{\bbP}{\mathbb P}
     % \newcommand{\bbE}{\mathbb E}
     % \newcommand{\bbN}{\mathbb N}
     % \newcommand{\sfE}{\mathsf E}
     % \newcommand{\calD}{\mathcal{D}}
     % % {{{ leqslant }}}
     % % slanted lower/greater equal signs
     % \renewcommand{\leq}{\leqslant}
     % \renewcommand{\le}{\leqslant}
     % \renewcommand{\geq}{\geqslant}
     % \renewcommand{\ge}{\geqslant}
     % % {{{ varepsilon }}}
     % \let\epsilon=\varepsilon
     % {{{ numberequationwithinsection }}}
     \numberwithin{equation}{section}
     % {{{ restate }}}
     % set of macros to deal with restating theorem environments (or anything
     % else with a label)
     % adapted from Boaz Barak
     \newcommand\MYcurrentlabel{xxx}
     % \MYstore{A}{B} assigns variable A value B
     \newcommand{\MYstore}[2]{%
     	\global\expandafter \def \csname MYMEMORY #1 \endcsname{#2}%
     }
     % \MYload{A} outputs value stored for variable A
     \newcommand{\MYload}[1]{%
     	\csname MYMEMORY #1 \endcsname%
     }
     % new label command, stores current label in \MYcurrentlabel
     \newcommand{\MYnewlabel}[1]{%
     	\renewcommand\MYcurrentlabel{#1}%
     	\MYoldlabel{#1}%
     }
     % new label command that doesn't do anything
     \newcommand{\MYdummylabel}[1]{}
     \newcommand{\torestate}[1]{%
     	% overwrite label command
     	\let\MYoldlabel\label%
     	\let\label\MYnewlabel%
     	#1%
     	\MYstore{\MYcurrentlabel}{#1}%
     	% restore old label command
     	\let\label\MYoldlabel%
     }
     \newcommand{\restatetheorem}[1]{%
     	% overwrite label command with dummy
     	\let\MYoldlabel\label
     	\let\label\MYdummylabel
     	\begin{theorem*}[Restatement of \cref{#1}]
     		\MYload{#1}
     	\end{theorem*}
     	\let\label\MYoldlabel
     }
     \newcommand{\restatelemma}[1]{%
     	% overwrite label command with dummy
     	\let\MYoldlabel\label
     	\let\label\MYdummylabel
     	\begin{lemma*}[Restatement of \cref{#1}]
     		\MYload{#1}
     	\end{lemma*}
     	\let\label\MYoldlabel
     }
     \newcommand{\restateprop}[1]{%
     	% overwrite label command with dummy
     	\let\MYoldlabel\label
     	\let\label\MYdummylabel
     	\begin{proposition*}[Restatement of \cref{#1}]
     		\MYload{#1}
     	\end{proposition*}
     	\let\label\MYoldlabel
     }
     \newcommand{\restatefact}[1]{%
     	% overwrite label command with dummy
     	\let\MYoldlabel\label
     	\let\label\MYdummylabel
     	\begin{fact*}[Restatement of \prettyref{#1}]
     		\MYload{#1}
     	\end{fact*}
     	\let\label\MYoldlabel
     }
     \newcommand{\restate}[1]{%
     	% overwrite label command with dummy
     	\let\MYoldlabel\label
     	\let\label\MYdummylabel
     	\MYload{#1}
     	\let\label\MYoldlabel
     }
     % {{{ mathabbreviations }}}
     \newcommand{\la}{\leftarrow}
     \newcommand{\sse}{\subseteq}
     \newcommand{\ra}{\rightarrow}
     \newcommand{\e}{\epsilon}
     \newcommand{\eps}{\epsilon}
     \newcommand{\eset}{\emptyset}
     % {{{ allowdisplaybreaks }}}
     % allows page breaks in large display math formulas
    % \allowdisplaybreaks
     % {{{ sloppy }}}
     % avoid math spilling on margin
     \sloppy
     % \newcommand*{\Id}{\mathrm{Id}}
     % \newcommand*{\sphere}{\mathbb{S}^{n-1}}
     % \newcommand*{\tr}{\mathrm{tr}}
     \newcommand*{\zo}{\{0,1\}}
     % \newcommand*{\bias}{\mathrm{bias}}
     % \newcommand*{\Sym}{\mathrm{sym}}
     % \newcommand*{\Perm}[2]{#1^{\underline{#2}}}
     % \newcommand*{\on}{\{\pm 1\}}
     % \newcommand*{\sos}{\mathrm{sos}}
     % \newcommand*{\U}{\mathcal{U}}
     % \newcommand*{\Normop}[1]{\Norm{#1}}
     % \newcommand*{\Lowner}{L\"owner\xspace}
     % \newcommand*{\normtv}[1]{\Norm{#1}_{\mathrm{TV}}}
     % \newcommand*{\normf}[1]{\Norm{#1}_{\mathrm{F}}}
     % \DeclareMathOperator{\diag}{diag}
      \DeclareMathOperator{\pE}{\tilde{\mathbb{E}}}
     % \DeclareMathOperator{\Span}{Span}
     % \DeclareMathOperator{\sspan}{Span}
     % \newcommand*{\tran}{^{\mkern-1.5mu\mathsf{T}}}
     \newcommand*{\tranpose}[1]{{#1}{}^{\mkern-1.5mu\mathsf{T}}}
     \newcommand*{\dyad}[1]{#1#1{}^{\mkern-1.5mu\mathsf{T}}}
     \newcommand{\1}{\bm{1}}
     
     \newcommand{\wh}{\widehat}
     \newcommand{\tmu}{\tilde{\mu}}
     \newcommand{\wt}{\mathsf{wt}}
     \newcommand{\cond}{\mathsf{cond}}
     \renewcommand{\S}{\mathbb{S}}
     \newcommand{\Sol}{\mathrm{Sol}}
     \newcommand{\In}{\mathcal{I}}
     \newcommand{\Ou}{\mathcal{O}}
     \newcommand{\Lin}{\mathrm{Lin}}
     
     \usepackage{etoolbox}
     \usepackage{refcount}
     \usepackage{footmisc}
     % temporary counter to save footnote number
     \newcounter{foottmpcnt}
     % custom \footnote command
     % #1: label
     % #2: footnote text
     \newcommand{\myfootnote}[2]{%
     	\csgdef{footnote@text@#1}{#2}%
     	\global\newcounter{footnote@page@#1}%
     	\setcounter{footnote@page@#1}{\value{page}}%
     	\footnote{\label{#1}#2}}
     % custom \footref command
     % #1: label
     \newcommand{\myfootref}[1]{%
     	\footref{#1}%
     	\ifnumcomp{\value{footnote@page@#1}}{=}{\value{page}}
     	{}
     	{\setcounter{foottmpcnt}{\value{footnote}}%
     		\setcounter{footnote}{\getrefnumber{#1}}%
     		\footnotetext{\csuse{footnote@text@#1}}%
     		\setcounter{footnote}{\value{foottmpcnt}}%
     		\setcounter{footnote@page@#1}{\value{page}}}}
     % TITLE
     

% to avoid loading the natbib package, add option nonatbib:
%     \usepackage[nonatbib]{neurips_2019}

\usepackage[utf8]{inputenc} % allow utf-8 input
\usepackage[T1]{fontenc}    % use 8-bit T1 fontst
\usepackage{hyperref}       % hyperlinks
\usepackage{url}            % simple URL typesetting
\usepackage{booktabs}       % professional-quality tables
\usepackage{amsfonts}       % blackboard math symbols
\usepackage{nicefrac}       % compact symbols for 1/2, etc.
\usepackage{microtype}      % microtypography

\title{List-decodeable Linear Regression}

% The \author macro works with any number of authors. There are two commands
% used to separate the names and addresses of multiple authors: \And and \AND.
%
% Using \And between authors leaves it to LaTeX to determine where to break the
% lines. Using \AND forces a line break at that point. So, if LaTeX puts 3 of 4
% authors names on the first line, and the last on the second line, try using
% \AND instead of \And before the third author name.

\author{%
  David S.~Hippocampus\thanks{Use footnote for providing further information
    about author (webpage, alternative address)---\emph{not} for acknowledging
    funding agencies.} \\
  Department of Computer Science\\
  Cranberry-Lemon University\\
  Pittsburgh, PA 15213 \\
  \texttt{hippo@cs.cranberry-lemon.edu} \\
  % examples of more authors
  % \And
  % Coauthor \\
  % Affiliation \\
  % Address \\
  % \texttt{email} \\
  % \AND
  % Coauthor \\
  % Affiliation \\
  % Address \\
  % \texttt{email} \\
  % \And
  % Coauthor \\
  % Affiliation \\
  % Address \\
  % \texttt{email} \\
  % \And
  % Coauthor \\
  % Affiliation \\
  % Address \\
  % \texttt{email} \\
}


\newcommand\blfootnote[1]{%
	\begingroup
	\renewcommand\thefootnote{}\footnote{#1}%
	\addtocounter{footnote}{-1}%
	\endgroup
}

\begin{document}

\maketitle

\begin{abstract}
We give the first polynomial-time algorithm for robust regression in the list-decodable setting where an adversary can corrupt a greater than $1/2$ fraction of examples. %outliers investigated in several recent works~\cite{DBLP:conf/stoc/CharikarSV17,KothariSteinhardt17,DBLP:conf/stoc/DiakonikolasKS18}.
For any $\alpha < 1$, our algorithm takes as input a sample $\{(x_i,y_i)\}_{i \leq n}$ of $n$ linear equations where $\alpha n$ of the equations satisfy $y_i = \langle x_i,\ell^*\rangle +\zeta$ for some small noise $\zeta$ and $(1-\alpha)n$ of the equations are {\em arbitrarily} chosen. It outputs a list $L$ of size $O(1/\alpha)$ - a fixed constant - that contains an $\ell$ that is close to $\ell^*$.

Our algorithm succeeds whenever the inliers are chosen from a \emph{certifiably} anti-concentrated distribution $D$. To  complement our result, we prove that the anti-concentration assumption on the inliers is information-theoretically necessary. As a corollary of our algorithmic result, we obtain a $(d/\alpha)^{O(1/\alpha^8)}$ time algorithm to find a $O(1/\alpha)$ size list when the inlier distribution is standard Gaussian. For discrete product distributions that are anti-concentrated only in \emph{regular} directions, we give an algorithm that achieves similar guarantee under the promise that $\ell^*$ has all coordinates of same magnitude.
%#distribution (and more generally, any spherically symmetric distribution with sub-exponential tails and linear-transformations thereof).


To solve the problem we introduce a new framework for list-decodable learning that strengthens the ``identifiability to algorithms'' paradigm based on the sum-of-squares method.

% In an independent work, Raghavendra and Yau~\cite{RY19} have obtained a similar result for list-decodable regression also using the sum-of-squares method
\end{abstract}

%!TEX root = ../main.tex

\section{Introduction}

%!TEX root = ../main.tex

In this work, we design algorithms for the problem of linear regression that are robust to training sets with an overwhelming ($\gg 1/2$) fraction of adversarially chosen outliers. 

Outlier-robust learning algorithms have been extensively studied (under the name \emph{robust statistics}) in mathematical statistics~\cite{MR0426989,maronna2006robust,huber2011robust,hampel2011robust}. However, the algorithms resulting from this line of work usually run in time exponential in the dimension of the data~\cite{bernholt2006robust}. An influential line of recent work \cite{journals/jmlr/KlivansLS09,journals/corr/AwasthiBL13, DBLP:journals/corr/DiakonikolasKKL16,DBLP:conf/focs/LaiRV16,DBLP:conf/stoc/CharikarSV17,KothariSteinhardt17,2017KS,HopkinsLi17,DBLP:journals/corr/DiakonikolasKK017a,DBLP:journals/corr/DiakonikolasKS17,DBLP:conf/colt/KlivansKM18} has focused on designing \emph{efficient} algorithms for outlier-robust learning. 

Our work extends this line of research. Our algorithms work in the \emph{list-decodable learning} framework. In this model, the majority of the training data (a $1 -\alpha$ fraction) can be adversarially corrupted leaving only an $\alpha \ll 1/2$ fraction of \emph{inliers}. Since uniquely recovering the underlying parameters is information-theoretically \emph{impossible} in such a setting, the goal is to output a list (with an absolute constant size) of parameters, one of which matches the ground truth. This model was introduced in~\cite{DBLP:conf/stoc/BalcanBV08} to give a discriminative framework for clustering. More recently, beginning with~\cite{DBLP:conf/stoc/CharikarSV17}, various works~\cite{DBLP:conf/stoc/DiakonikolasKS18,KothariSteinhardt17} have considered this as a model of \emph{untrusted} data. 

There has been a phenomenal progress in developing techniques for outlier-robust learning with a \emph{small} $(\ll 1/2)$-fraction of outliers (e.g. outlier \emph{filters}~\cite{DBLP:conf/focs/DiakonikolasKK016,DBLP:journals/corr/DiakonikolasKK017a}, separation oracles for inliers~\cite{DBLP:conf/focs/DiakonikolasKK016} or the \emph{sum-of-squares} method~\cite{2017KS,HopkinsLi17,KothariSteinhardt17,DBLP:conf/colt/KlivansKM18}). In contrast, progress on algorithms that tolerate the significantly harsher conditions in the list-decodable setting has been slower. The only prior works~\cite{DBLP:conf/stoc/CharikarSV17,DBLP:conf/stoc/DiakonikolasKS18,KothariSteinhardt17} in this direction designed list-decodable algorithms for mean estimation via somewhat \emph{ad hoc}, problem-specific methods. 

In this paper, we develop a principled technique to give the first efficient list-decodable learning algorithm for the fundamental problem of \emph{linear regression}. Our algorithm takes a corrupted set of linear equations with an $\alpha \ll 1/2$ fraction of inliers and outputs a $O(1/\alpha)$-size list of linear functions, one of which is guaranteed to be close to the ground truth (i.e., the linear function that correctly labels the inliers). Our key conceptual observation shows that list-decodable regression information-theoretically requires the inlier-distribution to be \emph{anti-concentrated}. Our algorithm succeeds whenever the distribution satisfies a stronger \emph{certifiable anti-concentration} condition. This class includes the standard gaussian distribution and more generally, any spherically symmetric distribution with strictly sub-exponential tails.

Prior to our work\footnote{There's a long line of work on robust regression algorithms (see for e.g. \cite{DBLP:conf/nips/Bhatia0KK17,conf/soda/KarmalkarP19}) that can tolerate corruptions only in the \emph{labels}. We are interested in algorithms robust against corruptions in both examples and labels.}, the state-of-the-art outlier-robust algorithms for linear regression~\cite{DBLP:conf/colt/KlivansKM18,conf/soda/DiakonikolasKS19,journals/corr/abs-1803-02815,journals/corr/abs-1802-06485} could handle only a small $(<0.1)$-fraction of outliers even under strong assumptions on the underlying distributions. 

List-decodable regression generalizes the well-studied~\cite{MR1028403,doi:10.1162/neco.1994.6.2.181,MR2757044,2013arXiv1310.3745Y,DBLP:journals/corr/BalakrishnanWY14,DBLP:conf/colt/ChenYC14,DBLP:conf/nips/Zhong0D16,DBLP:conf/aistats/SedghiJA16,DBLP:conf/colt/LiL18} and {\em easier} problem of \emph{mixed linear regression}: given $k$ ``clusters'' of examples that are labeled by one out of $k$ distinct unknown linear functions, find the unknown set of linear functions. All known techniques for the problem rely on faithfully estimating \emph{moment tensors} from samples and thus, cannot tolerate the overwhelming fraction of outliers in the list-decodable setting. On the other hand, since we can take any cluster as inliers and treat rest as outliers, our algorithm immediately yields new efficient algorithms for mixed linear regression. Unlike all prior works, our algorithms work without any pairwise separation or bounded condition-number assumptions on the $k$ linear functions. 








% The MLR model is a popular mixture model and has many applications due to its effectiveness in captur- ing non-linearity and its model simplicity [De Veaux, 1989, Jordan and Jacobs, 1994, Faria and Soromenho, 2010, Zhong et al., 2016]. It has also been a recent theoretical topic for analyzing benchmark algorithms for nonconvex optimization (e.g., [Chaganty and Liang, 2013, Klusowski et al., 2017]) or designing new algo- rithms (e.g., [Chen et al., 2014]). However, most of the existing works either restrict to very special settings (e.g., x of different components all from the standard Gaussian, or only k = 2 components) [Chen et al., 2014, Yi et al., 2014, Zhong et al., 2016, Balakrishnan et al., 2017, Klusowski et al., 2017], or have high sample or computational complexity far from optimal [Chaganty and Liang, 2013, Sedghi et al., 2016].



\paragraph{List-Decodable Learning via the Sum-of-Squares Method} Our algorithm relies on a strengthening of the robust-estimation framework based on the sum-of-squares (SoS) method. This paradigm has been recently used for clustering mixture models~\cite{HopkinsLi17,KothariSteinhardt17} and obtaining algorithms for moment estimation~\cite{2017KS} and linear regression~\cite{DBLP:conf/colt/KlivansKM18}
 relies on a strengthening of robust-estimation framework based on the sum-of-squares (SoS) method. This paradigm has been recently used for clustering mixture models~\cite{HopkinsLi17,KothariSteinhardt17} and obtaining algorithms for moment estimation~\cite{2017KS} and linear regression~\cite{DBLP:conf/colt/KlivansKM18} that are resilient to a small $(\ll 1/2)$ fraction of outliers under the mildest known assumptions on the underlying distributions. Ths method reduces outlier-robust algorithm design to finding ``simple'' proofs of  unique \emph{identifiability} of the unknown parameter of the original distribution from a corrupted sample. However, this principled method works only in the setting with a small ($\ll 1/2$) fraction of outliers. As a consequence, the work of~\cite{KothariSteinhardt17} for mean estimation in the list-decodable setting relied on ``supplementing'' the SoS method with a somewhat \emph{ad hoc}, problem-dependent technique. 

As an important conceptual contribution, our work yields a framework for list-decodable learning that recovers some of the simplicity of the general blueprint. To do this, we give a general method for \emph{rounding pseudo-distributions} in the setting with $\gg 1/2$ fraction outliers. A key step in our rounding builds on the work of~\cite{KS19} who developed such a method to give a simpler proof of the list-decodable mean estimation result of~\cite{KothariSteinhardt17}. In Section~\ref{sec:overview}, we explain our ideas in detail. 

The results in all the works above hold whenever the underlying distribution satisfies a certain \emph{certified concentration} condition formulated within the SoS system via higher moment bounds. An important contribution of this work is formalizing an \emph{anti-concentration} condition within the SoS system. Unlike the bounded moment condition, there is no canonical phrasing  within SoS for such statements. We choose a form that allows proving ``certified anti-concentration'' for a distribution by showing the existence of a certain approximating polynomial. This allows showing certified anti-concentration of natural distributions via a completely modular approach that relies on a beautiful line of works that construct ``weighted '' polynomial approximators~\cite{2007math......1099L}. 

We believe that our framework for list-decodable estimation and our formulation of certified anti-concentration condition will likely have further applications in outlier-robust learning. 
% We focus on the setting where the noise rate is larger than $1/2$ and unique recovery of the underlying parameters is information-theoretically {\em impossible}.  We study the supervised learning problem of linear regression and assume that a learner has been given a training set where a (possibly greater than $1/2$) fraction of examples have been {\em arbitrarily} corrupted in both the labels and locations of the data points.  Given this training set as input, we give the first efficient algorithm for outputting a {\em list} of linear functions, one of which agrees with the original, uncorrupted data set (the list is {\em constant-size} where the constant depends on the noise rate).  We refer to this problem as {\em list-decodable linear regression}\footnote{Our setting is similar in spirit to list-decodable error correcting codes, but the problem statements and techniques are quite different.}.  It is easy to see that list-decodable linear regression is harder than classical {\em mixed linear regression}, an intensely studied problem in machine learning.  Our algorithm requires an assumption on the marginal distribution, namely {\em certifiable anti-concentration}, which we prove is information-theoretically necessary for list-decodability in this setting.  
\subsection{Our Results}
We first define our model for generating samples for list-decodable regression.

\begin{model}[Robust Linear Regression]
For $0 <\alpha < 1$ and $\ell^* \in \R^d$ with $\|\ell^*\|_2 \leq 1$, let $\Lin_D(\alpha,\ell^*)$ denote the following probabilistic process to generate $n$ noisy linear equations $\cS = \{ \langle x_i, a \rangle = y_i\mid 1\leq i \leq n\}$ in variable $a \in \R^d$ with $\alpha n$ \emph{inliers} $\cI$ and $(1-\alpha)n$ \emph{outliers} $\cO$:
\begin{enumerate}
\item Construct $\cI$ by choosing $\alpha n$ i.i.d. samples $x_i \sim D$ and set $y_i = \langle x_i,\ell^* \rangle + \zeta$ for additive and independent noise $\zeta$,
\item Construct $\cO$ by  choosing the remaining $(1-\alpha)n$ equations arbitrarily and potentially adversarially w.r.t the inliers $\cI$.
\end{enumerate}
\label{model:random-equations}
\end{model}
The bound on the norm of $\ell^*$ is without any loss of generality. 
% Our algorithm naturally extends to handle additive noise in the labels. 
% We will focus mostly on the noiseless case for clarity of exposition.
Note that $\alpha$ is measure of the ``signal'' (fraction of inliers) and can be $\ll 1/2$. 

An $\eta$-approximate algorithm for list-decodable regression takes input a sample from $\Lin_D(\alpha,\ell^*)$ and outputs a \emph{constant} (depending only on $\alpha$) size list $L$ of linear functions such that there is some $\ell \in L$ that is $\eta$-close to $\ell^*$.

One of our key conceptual contributions is to identify the strong relationship between \emph{anti-concentration inequalities} and list-decodable regression. Anti-concentration inequalities are well-studied~\cite{ErdosLittlewoodOfford,MR2965282-Tao12,MR2407948-Rudelson08} in probability theory and combinatorics. The simplest of these inequalities upper bound the probability that a high-dimensional random variable has zero projections in any direction.

\begin{definition}[Anti-Concentration]
A $\R^d$-valued zero-mean random variable $Y$ has a $\delta$-\emph{anti-concentrated} distribution if $\Pr[ \iprod{Y,v}=0 ]< \delta$. 
\end{definition}

In Proposition~\ref{prop:identifiability}, we provide a simple but conceptually illuminating proof that anti-concentration is \emph{sufficient} for list-decodable regression. In Theorem~\ref{thm:main-lower-bound}, we prove a sharp converse and show that anti-concentration is information-theoretically \emph{necessary} for even noiseless list-decodable regression. This lower bound surprisingly holds for a natural distribution: uniform distribution on $\zo^d$ and more generally, uniform distribution on $[q]^d$ for  $q = \{0,1,2\ldots,q\}$.

\begin{theorem}[See Proposition~\ref{prop:identifiability} and Theorem~\ref{thm:main-lower-bound}]
There is a (inefficient) list-decodable regression algorithm for $\Lin_D(\alpha,\ell^*)$ with list size $O(\frac{1}{\alpha})$ whenever $D$ is $\alpha$-anti-concentrated. 
Further, there exists a distribution $D$ on $\R^d$ that is $\paren{\alpha+\epsilon}$-anti-concentrated for every $\epsilon >0$ but there is no algorithm for $\frac{\alpha}{2}$-approximate list-decodable regression for $\Lin_D(\alpha,\ell^*)$  that returns a list of size $<d$. 
\end{theorem}
For our efficient algorithms, we need a \emph{certified} version of the anti-concentration condition. 
To handle additive noise of variance $\zeta^2$, we need a control of $\Pr[ |\iprod{x,v}| \leq \zeta]$. 
Thus, we extend our notion of anti-concentration and then define a \emph{certified} analog of it:
\begin{definition}[Certifiable Anti-Concentration] \label{def:certified-anti-concentration}
A random variable $Y$ has a $k$-\emph{certifiably} $(C,\delta)$-anti-concentrated distribution if there is a univariate polynomial $p$ satisfying $p(0) = 1$ such that there is a degree $k$ sum-of-squares proof of the following two inequalities: 
\begin{enumerate} 
\item $\forall v$, $\langle Y,v\rangle^2 \leq \delta^2 \E \langle Y,v\rangle^2$ implies $(p(\langle Y,v\rangle) -1)^2\leq \delta^2$.
\item $\forall v$, $\|v\|_2^2 \leq 1$ implies  $\E p^2(\left \langle Y,v\rangle \right) \leq C\delta$.
\end{enumerate}
\end{definition} 
Intuitively, certified anti-concentration asks for a \emph{certificate} of the anti-concentration property of $Y$ in the ``sum-of-squares'' proof system (see Section~\ref{sec:preliminaries} for precise definitions). SoS is a proof system that reasons about  polynomial inequalities. Since the ``core indicator'' $\1(|\iprod{x,v}| \leq \delta)$ is not a polynomial, we phrase the condition in terms of an approximating polynomial $p$.
\Pnote{seems like it would be nice to have some explanation for the properties of the polynomial asked for in this definition...}
We are now ready to state our main result.
\begin{restatable}[List-Decodable Regression]{theorem}{main} \label{thm:main}
For every $\alpha, \eta > 0$ and a $k$-certifiably $(C,\alpha^2 \eta^2/10C)$-anti-concentrated distribution $D$ on $\R^d$, there exists an algorithm that takes input a sample generated according to $\Lin_D(\alpha,\ell^*)$ and outputs a list $L$ of size $O(1/\alpha)$ such that there is an $\ell \in L$ satisfying $\| \ell - \ell^*\|_2 < \eta$ with probability at least $0.99$ over the draw of the sample. The algorithm needs a sample of size $n = (kd)^{O(k)}$ and runs in time $n^{O(k)} = (kd)^{O(k^2)}$.
\end{restatable} 
\begin{remark}[Tolerating Additive Noise]
For additive noise of variance $\zeta^2$ in the inlier labels, our algorithm, in the same running time and sample complexity, outputs a list of size $O(1/\alpha)$ that contains an $\ell$ satisfying $\|\ell-\ell^*\|_2 \leq \frac{\zeta}{\alpha} + \eta$. Since we normalize $\ell^*$ to have unit norm, this guarantee is meaningful only when $\zeta \ll \alpha$. %It is not clear if better noise-tolerance is possible for list-decodable regression.
\end{remark}

\begin{remark}[Exponential Dependence on $1/\alpha$]
List-decodable regression algorithms immediately yield algorithms for mixed linear regression (MLR) without any assumptions on the components. The state-of-the-art algorithm for MLR with gaussian components~\cite{DBLP:conf/colt/LiL18} has an exponential dependence on $k=1/\alpha$ in the running time in the absence of strong pairwise separation or small condition number of the components. Liang and Liu~\cite{DBLP:conf/colt/LiL18} (see Page 10) use the relationship to learning mixtures of $k$ gaussians (with an $\exp(k)$ lower bound~\cite{DBLP:conf/focs/MoitraV10}) to note that algorithms with polynomial dependence on $1/\alpha$ for MLR and thus, also for list-decodable regression might not exist. 
\end{remark}


\paragraph{Certifiably anti-concentrated distributions} In Section~\ref{sec:certified-anti-concentration}, we show certifiable anti-concentration of some well-studied families of distributions. This includes the standard gaussian distribution and more generally any anti-concentrated spherically symmetric distribution with strictly sub-exponential tails. We also show that simple operations such as scaling, applying well-conditioned linear transformations and sampling preserve certifiable anti-concentration. This yields:
\begin{corollary}[List-Decodable Regression for Gaussian Inliers]
For every $\alpha, \eta > 0$ there's an algorithm for list-decodable regression for the model $\Lin_D(\alpha,\ell^*)$ with $D = \cN(0,\Sigma)$ with $\lambda_{\max}(\Sigma)/\lambda_{min}(\Sigma) = O(1)$ that needs $n = (d/\alpha \eta)^{O\left(\frac{1}{\alpha^4 \eta^4}\right)}$  samples and runs in time $n^{O\left(\frac{1}{\alpha^4 \eta^4}\right)} = (d/\alpha \eta)^{O\left(\frac{1}{\alpha^8 \eta^8}\right)}$.
\end{corollary} 

We note that certifiably anti-concentrated distributions are more restrictive compared to the families of distributions for which the most general robust estimation algorithms work~\cite{2017KS,KothariSteinhardt17,DBLP:conf/colt/KlivansKM18}. To a certain extent, this is inherent. The families of distributions considered in these prior works do not satisfy anti-concentration in general.  And as we discuss in more detail in Section~\ref{sec:overview}, anti-concentration is information-theoretically \emph{necessary} (see Theorem~\ref{thm:main-lower-bound}) for list-decodable regression. This surprisingly rules out families of distributions that might appear natural and ``easy'', for example, the uniform distribution on $\zo^n$. In fact, our lower bound shows the impossibility of even the ``easier'' problem of mixed linear regression on this distribution.

We rescue this to an extent for the special case when $\ell^*$ in the model $\Lin(\alpha,\ell^*)$ is a "Boolean vector", i.e., has all coordinates of equal magnitude. Intuitively, this helps because  while the the uniform distribution on $\zo^n$ (and more generally, any discrete product distribution) is badly anti-concentrated in sparse directions, they are well anti-concentrated~\cite{ErdosLittlewoodOfford} in the directions that are far from any sparse vectors. 

As before, for obtaining efficient algorithms, we need to work with a \emph{certified} version (see Definition~\ref{def:certified-anti-concentration-Boolean}) of such a restricted anti-concentration condition. As a specific Corollary (see Theorem~\ref{thm:Booleanmain} for a more general statement), this allows us to show:
\begin{theorem}[List-Decodable Regression for Hypercube Inliers] \label{thm:boolcube}
For every $\alpha, \eta > 0$ there's an $\eta$-approximate algorithm for list-decodable regression for the model $\Lin_D(\alpha,\ell^*)$ with $D$ is uniform on $\zo^d$ that needs $n = (d/\alpha \eta)^{O(\frac{1}{\alpha^4 \eta^4})}$  samples and runs in time $n^{O(\frac{1}{\alpha^4 \eta^4})} = (d/\alpha \eta)^{O(\frac{1}{\alpha^8 \eta^8})}$.
\end{theorem} 

In Section~\ref{sec:hypercube}, we obtain similar results for general product distributions. It is an important open problem to prove certified anti-concentration for a broader family of distributions. % In particular, this yields a $d^{O(\frac{1}{\alpha^2 \eta^2})}$-sample and $d^{O(\frac{1}{\alpha^4 \eta^4})}$-time algorithm for list-decodable regression when $D$ is the uniform distribution on the Boolean hypercube/solid hypercube. 


\subsection{Concurrent Work}
% % The model of {\em list-decodable} learning was introduced by Balcan et al. \cite{DBLP:conf/stoc/BalcanBV08} and recent works have given the first efficient list-decodable learning algorithms for {\em unsupervised} learning problems such as mean estimation and learning mixture models~\cite{KothariSteinhardt17,DBLP:conf/stoc/DiakonikolasKS18,DBLP:conf/stoc/CharikarSV17}.  
% While solving linear regression in various noise models is a heavily studied topic, most papers consider the setting where only the labels have been corrupted (see e.g., \cite{DBLP:conf/nips/Bhatia0KK17,conf/soda/KarmalkarP19}).  %The first efficient algorithms for linear regression in the robust setting (where a fraction of the both the labels and locations of the points can be arbitrarily corrupted) were only recently given by \cite{DBLP:conf/colt/KlivansKM18,conf/soda/DiakonikolasKS19,journals/corr/abs-1803-02815,journals/corr/abs-1802-06485} under various assumptions on the marginal distribution.  Those works require the noise rate to be less than a small constant (roughly $.1$).  
In an independent and concurrent work, Raghavendra and Yau have given similar results for list-decodable linear regression and also use the sum-of-squares paradigm \cite{RY19}.

% \subsection{Our Techniques}

% We emphasize two key technical innovations of this work.  The first is a general framework for list-decodable learning using the sum-of-squares approach (a prior work by Kothari and Steinhardt \cite{KothariSteinhardt17} used a sum of squares approach for list-decodable learning of mixture models, but the solution is highly specialized for their particular application).   

% More concretely, we introduce a {\em general method} of rounding SDP solutions in the high noise regime: we associate a weight variable to each training point and interpret a particular point's weight as a "vote" for its guess of the unknown linear function.  This induces a probability distribution on linear functions and repeatedly sampling from this distribution results in our list of candidate solutions. 

% Our second contribution is a translation of the notion of anti-concentration into the sum-of-squares proof system.  All recent work on robust learning requires a bounded moment condition on the marginal distribution, and this is straightforward to encode using polynomial inequalities.  Encoding anti-concentration, however, is more challenging.  We use a custom polynomial approximator for an indicator function on the real line, and show that the required properties of this polynomial can be expressed via sum-of-squares. 





%!TEX root = ../main.tex



\section{Overview of our Technique} \label{sec:overview}
In this section, we illustrate the important ideas in our algorithm for list-decodable regression. 
Thus, given a sample $\cS = \{(x_i,y_i)\}_{i = 1}^n$ from $\Lin_D(\alpha,\ell^*)$, we must construct a constant-size list $L$ of linear functions containing an $\ell$ close to $\ell^*$. 

Our algorithm is based on the sum-of-squares method. We build on the ``identifiability to algorithms'' paradigm developed in several prior works~\cite{DBLP:conf/colt/BarakM16,MR3388192-Barak15,DBLP:conf/focs/MaSS16,2017KS,HopkinsLi17,KothariSteinhardt17,DBLP:conf/colt/KlivansKM18} with some important conceptual differences. 

\paragraph{An \emph{inefficient} algorithm} Let's start by designing an inefficient algorithm for the problem. This may seem simple at the outset. But as we'll see, solving this relaxed problem will rely on some important conceptual ideas that will serve as a starting point for our efficient algorithm. 

Without computational constraints, it is natural to just return the list $L$ of all linear functions $\ell$ that correctly labels all examples in some $S \subseteq \cS$ of size $\alpha n$. We call such an $S$, a large, \emph{soluble} set. True inliers $\cI$ satisfy our search criteria so $\ell^* \in L$. However, it's not hard to show (Proposition~\ref{prop:brute-force-doesn't-work}) that one can choose outliers so that the list so generated has size $\exp(d)$ (far from a fixed constant!).

A potential fix is to search instead for a \emph{coarse soluble partition} of $\cS$, if it exists, into disjoint $S_1, S_2,\ldots, S_k$ and  linear functions $\ell_1, \ell_2, \ldots, \ell_k$ so that every $|S_i| \geq \alpha n$ and $\ell_i$ correctly computes the labels in $S_i$. In this setting, our list is small ($k\leq 1/\alpha$). But it is easy to construct samples $\cS$ for which this fails 
because there are coarse soluble partitions of $\cS$ where every $\ell_i$ is far from $\ell^*$. %Call such a partition a coarse partition. 
% Nevertheless, samples that do, occur to the well-studied setting of \emph{mixed linear regression}.     
\paragraph{Anti-Concentration} 
It turns out that any (even inefficient) algorithm for list-decodable regression provably (see Theorem~\ref{thm:main-lower-bound}) \emph{requires} that the distribution of inliers\footnote{As in the standard robust estimation setting, the outliers are  arbitrary and potentially adversarially chosen.} be sufficiently \emph{anti-concentrated}:

% For technical reasons our proxy for non-degeneracy will be \emph{anti-concentration}: \Pnote{I think anti-concentration as in the definition below is exactly the notion of non-degeneracy we need (and not just a proxy).}

\begin{definition}[Anti-Concentration]
A $\R^d$-valued random variable $Y$ with mean $0$ is $\delta$-anti-concentrated\footnote{Definition~\ref{def:certified-anti-concentration} differs slightly to handle list-decodable regression with additive noise in the inliers.} if for all non-zero $v$, $\Pr[ \iprod{Y,v} = 0 ] < \delta$. A set $T \subseteq \R^d$ is $\delta$-anti-concentrated if the uniform distribution on $T$ is $\delta$-anti-concentrated.
\end{definition}


 
% % Our technique relies on side-stepping the issues in algorithm design by focusing just on giving a proof of \emph{identifiability}. 
% % Operationally, this means that we use a (large-enough) input sample to give a procedure that can \emph{verify} purported solutions.



% % \emph{ Given a list $L$ of linear functions, how can we ascertain that there is an $\ell \in L$ close to $\ell^*$?} 


% By applying this test to all possible lists of constant size, we can obtain a good solution establishing identifiability. 

% Perhaps the first idea here is to $\ell \in L$, there be a $S_{\ell} \subseteq \cS$ of size $\alpha n$ such that $\ell$ correctly computed the labels of all examples in $S_{\ell}$. The $S_{\ell}$s clearly serve as a ``certificate'' that every linear function $\ell \in L$ indeed satisfies some $\alpha$ fraction of the equations in $\cS$. 





% A natural approach is to look at $\Sol$, the set of \emph{all} pairs $(S,\ell)$ such that $S \subseteq [n]$ of size $|S| = \alpha n$ and $\iprod{x_i,\ell} = y_i$ for every $i \in S$.  
% We can simply hope to return the list $L$ of $\ell$s that appear in $\Sol$. 
% Since $(\cI,\ell^*)$ is contained in $\Sol$, $\ell^* \in L$. 
% However, it's not hard to show (see Lemma~\ref{}) this list $L$ can be $\exp(\Omega(n))$ in size - far from the absolute constant size we are shooting for.

% Indeed, any solution to list-decodable regression must (at least implicitly) establish $\Sol$ can be pruned down to an absolute constant size without affecting $(\cI,\ell^*)$.
% The key idea that is both necessary (see Lemma~\ref{}) and sufficient to establish the \emph{existence} of a small list-decoding of the sample is the anti-concentration property of $\cI$.
% In Lemma~\ref{}, we show that anti-concentration of $\cI$ is also information-theoretically \emph{necessary} for list-decodable regression). 
As we discuss next, anti-concentration is also \emph{sufficient} for list-decodable regression. Intuitively, this is because anti-concentration of the inliers prevents the existence of a soluble set that intersects significantly with $\cI$ and yet can be labeled correctly by $\ell \neq \ell^*$. This is simple to prove in the special case when $\cS$ admits a coarse soluble partition. 

% Let's start with the simple proof in the special case when $\cS$ admits a soluble partition. %Thus, our second idea above immediately gives us an inefficient algorithm for list-decodable regression whenever $\cS$ admits a soluble partition. 



\begin{proposition}
Suppose $\cI$ is $\alpha$-anti-concentrated. Suppose there exists a partition $S_1, S_2,\ldots, S_k \subseteq \cS$ such that each $|S_i| \geq \alpha n$ and there exist $\ell_1, \ell_2, \ldots, \ell_k$ such that $y_j = \iprod{\ell_i, x_j}$ for every $j \in S_i$. Then, there is an $i$ such that $\ell_i = \ell^*$. \label{prop:simple-uniqueness-partition}
\end{proposition}

\begin{proof}
Since $k \leq 1/\alpha$, there is a $j$ such that $|\cI \cap S_j| \geq \alpha |\cI|$. 
Then, $\iprod{x_i, \ell_j}= \iprod{x_i, \ell^*}$ for every $i \in \cI \cap S_j$. 
Thus, $\Pr_{i \sim \cI}[\iprod{x_i,\ell_j-\ell^*} = 0]\geq \alpha$. This contradicts anti-concentration of $\cI$ unless $\ell_j - \ell^* = 0$.
\end{proof}

%Observe that \emph{any} soluble partition for $L$ allows us to conclude that $\ell* \in L$. 
The above proposition allows us to use \emph{any} soluble partition as a \emph{certificate} of correctness for the associated list $L$. Two aspects of this certificate were crucial in the above argument: 1) \emph{largeness}: each $S_i$ is of size $\alpha n$ - so the generated list is small, and, 2) \emph{uniformity}: every sample is used in exactly one of the sets so $\cI$ must intersect one of the $S_i$s in at least $\alpha$-fraction of the points. 

\paragraph{Identifiability via anti-concentration} For arbitrary $\cS$, a coarse soluble partition might not exist. So we will generalize coarse soluble partitions to obtain certificates that exist for every sample $\cS$ and guarantee largeness and a relaxation of uniformity (formalized below). For this purpose, it is convenient to view such certificates as distributions $\mu$ on $\geq \alpha n$ size soluble subsets of $\cS$ so any collection $\cC\subseteq 2^{\cS}$ of $\alpha n$ size sets corresponds to the uniform distribution $\mu$ on  $\cC$. 

To precisely define uniformity, let $W_i(\mu) = \E_{S \sim \mu} [ \1(i \in S)]$ be the ``frequency of i'', that is, probability that the $i$th sample is chosen to be in a set drawn according to $\mu$. Then, the uniform distribution $\mu$ on any coarse soluble $k$-partition satisfies $W_i = \frac{1}{k}$ for every $i$. That is, all samples $i \in \cS$ are \emph{uniformly} used in such a $\mu$. To generalize this idea, we define $\sum_i W_i(\mu)^2$ as the \emph{distance to uniformity} of $\mu$. Up to a shift, this is simply the variance in the frequencies of the points in $\cS$ used in draws from $\mu$. Our generalization of a coarse soluble partition of $\cS$ is any $\mu$ that minimizes $\sum_i W_i(\mu)^2$, the distance to uniformity, and is thus \emph{maximally uniform} among all distributions supported on large soluble sets. Such a $\mu$ can be found by convex programming. % coarse soluble partition, it  %The uniform distribution on any soluble $k$-partition of $\cS$ has maximum possible uniformity ($\forall i$, $W_i = \frac{1}{k}$) when it exists.  
%However, since we require $\cD$ to be supported on soluble sets, such a partition may not exist and there may be a non-trivial choice of a maximally fair distribution. 

% We can now precisely define maximally fair distribution $\cD$: we say that a distribution $\cD$ over $\alpha n$ size \emph{soluble} subsets of $\cS$ is \emph{maximally} fair if for every for every distribution $\cD'$ on $\alpha n$-size soluble sets, $\sum_i W_i(\cD)^2 \leq \sum_i W_i(\cD')^2$. Observe that if $\cS$ admits a soluble partition into $\alpha n$ size sets, then, the uniform distribution on the sets in the partition is always maximally fair. 

% It turns out that if we replace a soluble partition by a maximal distribution $\cD$ over soluble sets of size $\alpha n$, our argument from before goes through as is. 
 
The following claim generalizes Proposition~\ref{prop:simple-uniqueness-partition} to derive the same conclusion starting from any maximally uniform distribution supported on large soluble sets.

\begin{proposition} \label{prop:fair-weight}
For a maximally uniform $\mu$ on $\alpha n$ size soluble subsets of $\cS$, $\sum_{i \in \cI} \E_{S \sim \mu} [\1 \Paren{i \in S}] \geq \alpha |\cI|$. 
\end{proposition}
The proof proceeds by contradiction (see Lemma~\ref{lem:large-weight-on-inliers}). We show that if $\sum_{i \in \cI} W_i(\mu) \leq \alpha |\cI|$, then we can strictly reduce the distance to uniformity by taking a mixture of $\mu$ with the distribution that places all its probability mass on $\cI$. This allow us to obtain an (inefficient) algorithm for list-decodable regression establishing identifiability. 
\begin{proposition}[Identifiability for List-Decodable Regression] \label{prop:identifiability}
Let $\cS$ be sample from $\Lin(\alpha,\ell^*)$ such that $\cI$ is $\delta$-anti-concentrated for $\delta < \alpha$. Then, there's an (inefficient) algorithm that finds a list $L$ of size $\frac{20}{\alpha-\delta}$ such that $\ell^* \in L$ with probability at least $0.99$.
\end{proposition}
\begin{proof}
Let $\mu$ be \emph{any} maximally uniform distribution over $\alpha n$ size soluble subsets of $\cS$. 
For $k = \frac{20}{\alpha-\delta}$, let $S_1, S_2, \ldots, S_k$ be independent samples from $\mu$.
Output the list $L$ of $k$ linear functions that correctly compute the labels in each $S_i$.

To see why $\ell^* \in L$, observe that $\E |S_j \cap \cI|= \sum_{i \in \cI} \E \1(i \in S_j) \geq \alpha |\cI|$. 
By averaging, $\Pr [|S_j \cap \cI| \geq \frac{\alpha+\delta}{2} |\cI|] \geq \frac{\alpha-\delta}{2}$. Thus, there's  a $j \leq k$ so that $|S_j \cap \cI| \geq \frac{\alpha+\delta}{2} |\cI|$ with probability at least $1-(1-\frac{\alpha-\delta}{2})^{\frac{20}{\alpha-\delta}} \geq 0.99$. We can now repeat the argument in the proof of Proposition~\ref{prop:simple-uniqueness-partition} to conclude that any linear function that correctly labels $S_j$ must equal $\ell^*$.
\end{proof}


\paragraph{An efficient algorithm}
Our identifiability proof suggests the following simple algorithm: 1) find \emph{any} maximally uniform distribution $\mu$ on soluble subsets of size $\alpha n$ of $\cS$, 2) take $O(1/\alpha)$ samples $S_i$ from $\mu$ and 3) return the list of linear functions that correctly label the equations in $S_i$s. This is inefficient because searching over distributions is NP-hard in general. 

To make this into an efficient algorithm, we start by observing that soluble subsets $S \subseteq \cS$ of size $\alpha n$ can be described by the following set of quadratic equations where $w$ stands for the indicator of $S$ and $\ell$, the linear function that correctly labels the examples in $S$. 

\begin{equation} \label{eq:quadratic-formulation}
  \cA_{w,\ell}\colon
  \left \{
    \begin{aligned}
      &&
      \textstyle\sum_{i=1}^n w_i
      &= \alpha n\\
      &\forall i\in [n].
      & w_i^2
      & =w_i \\
      &\forall i\in [n].
      & w_i \cdot (y_i - \iprod{x_i,\ell})
      & = 0\\
      &
      &\|\ell\|^2
      & \leq 1\\
    \end{aligned}
  \right \}
\end{equation} 

Our efficient algorithm searches for a maximally uniform \emph{pseudo-distribution} on $w$ satisfying \eqref{eq:quadratic-formulation}. Degree $k$ pseudo-distributions (see Section~\ref{sec:preliminaries} for precise definitions) are generalization of distributions that nevertheless ``behave'' just as distributions whenever we take (pseudo)-expectations (denoted by $\pE$) of a class of degree $k$ polynomials. And unlike distributions, degree $k$ pseudo-distributions satisfying\footnote{See Fact~\ref{fact:eff-pseudo-distribution} for a precise statement.} polynomial constraints (such as \eqref{eq:quadratic-formulation}) can be computed in time $n^{O(k)}$. 

For the sake of intuition, it might be helpful to (falsely) think of pseudo-distributions $\tmu$ as simply distributions where we only get access to moments of degree $\leq k$. Thus, we are allowed to compute expectations of all degree $\leq k$ polynomials with respect to $\tmu$. Since $W_i(\tmu) = \pE_{\tmu} w_i$ are just first moments of $\tmu$, our notion of maximally uniform distributions extends naturally to pseudo-distributions. This allows us to prove an analog of Proposition~\ref{prop:fair-weight} for pseudo-distributions and gives us an efficient replacement for Step 1.

\begin{proposition}
For any maximally uniform $\tmu$ of degree $\geq 2$,  $\sum_{i \in \cI} \pE_{\tmu}[w_i]  \geq \alpha |\cI| = \alpha \sum_{i\in [n]} \pE_{\tmu}[w_i]$\mper\label{prop:good-weight-on-inliers}
\end{proposition} 

For Step 2, however, we hit a wall: it's not possible to obtain independent samples from $\tmu$ given only low-degree moments. Our algorithm relies on an alternative strategy instead. 

Consider the vector $v_i = \frac{\pE_{\tmu}[w_i \ell]}{\pE_{\tmu}[w_i]}$ whenever $\pE_{\tmu}[w_i] \neq 0$ (set $v_i$ to zero, otherwise).  This is simply the (scaled) average, according to $\tmu$, of all the linear functions $\ell$ that are used to label the sets $S$ of size $\alpha n$ in the support of $\tmu$ whenever $i \in S$. Further, $v_i$ depends only on the first two moments of $\tmu$.

We think of $v_i$s as ``guesses''%
%\footnote{A similar idea was first observed in~\cite{KS19} to obtain a simpler algorithm for list-decodable robust mean estimation.}
made by the $i$th sample for the unknown linear function. 
Let us focus our attention on the guesses $v_i$ of $i \in \cI$ - the inliers. We will show that according to the distribution proportional to $\pE[w]$, the average squared distance of $v_i$ from $\ell^*$ is at max $\eta$:

\begin{equation}
\frac{1}{\sum_{i \in \cI} \pE[w_i]} \sum_{i \in \cI} \pE[w_i] \| v_i - \ell^*\|_2  < \eta \mper \label{eq:inliers-guess-well}\tag{$\star$}
\end{equation}

Before diving into \eqref{eq:inliers-guess-well}, let's see how it gives us our efficient list-decodable regression algorithm:

\begin{enumerate}
	\item Find a pseudo-distribution $\tmu$ satisfying \eqref{eq:quadratic-formulation} that minimizes distance to uniformity $\sum_i \pE_{\tmu}[w_i]^2$.
	\item For $O(\frac{1}{\alpha})$ times, independently choose a random index $i \in [n]$ with probability proportional to $\pE_{\tmu}[w_i]$ and return the list of corresponding $v_i$s. 
\end{enumerate} 

Step 1 above is a convex program and can be solved in polynomial time. Let's analyze step 2 to see why the algorithm works. Using \eqref{eq:inliers-guess-well} and Markov's inequality, conditioned on $i \in \cI$, $\|v_i - \ell^*\|_2 \leq 2 \eta$ with probability $\geq 1/2$. By Proposition~\ref{prop:good-weight-on-inliers}, $\frac{\sum_{i \in \cI} \pE[w_i]}{\sum_{i \in [n] \pE[w_i]}} \geq \alpha$ so $i \in \cI$ with probability at least $\alpha$. Thus in each iteration of step 2, with probability at least $\alpha/2$, we choose an $i$ such that $v_i$ is $2\eta$-close to $\ell^*$. Repeating $O(1/\alpha)$ times gives us the $0.99$ chance of success.

\paragraph{\eqref{eq:inliers-guess-well} via anti-concentration} As in the information-theoretic argument, \eqref{eq:inliers-guess-well} relies on the anti-concentration of $\cI$.
Let's do a quick proof for the case when $\tmu$ is an actual distribution $\mu$.

\begin{proof}[Proof of \eqref{eq:inliers-guess-well} for actual distributions $\mu$]%\Pnote{we already defined ws when we wrote the system 2.1. So changed the language here to "observe that...".}
Observe that $\mu$ is a distribution over $(w,\ell)$ satisfying \eqref{eq:quadratic-formulation}. Recall that $w$ indicates a subset $S \subseteq \cS$ of size $\alpha n$ and $w_i = 1$ iff $i \in S$. And $\ell \in \R^d$ satisfies all the equations in $S$.

By Cauchy-Schwarz, $\sum_i \|\E[w_i \ell] - \E[w_i] \ell^*\| \leq \E_{\mu} [\sum_{i \in \cI} w_i \|\ell - \ell^*\|]$. 
Next, as in Proposition~\ref{prop:simple-uniqueness-partition}, since $\cI$ is $\eta$-anti-concentrated, and for all $S$ such that $|\cI \cap S| \geq \eta |\cI|$,  $\ell-\ell^*= 0$. Thus, any such $S$ in the support of $\mu$ contributes $0$ to the expectation above. We will now show that the contribution from the remaining terms is upper bounded by $\eta$. Observe that since $\|\ell-\ell^*\| \leq 2$, \\$\E [\sum_{i \in \cI} w_i \|\ell - \ell^*\|] = \E [\1\Paren{|S \cap \cI|< \eta |\cI|}w_i \|\ell - \ell^*\|] = \E [\sum_{i \in S \cap \cI} \|\ell - \ell^*\|]  \leq 2\eta |\cI|$.
\end{proof}



% To gain intuition about the distributions $\tmu$ for which such a statement might be true - let us imagine two extreme situations. First, assume that $\tmu$ is such that for every $(w,\ell)$ in the support, whenever $w_i = 1$ for some $i \in \cI$, then, $w$ is in fact the indicator of $\cI$. That is, if the subset indicated by $w$ manages to include even a single point from $\cI$, then it in fact indicates the subset $\cI$ itself. In this case, observe that $v_i = \ell^*$(!). 

% To understand a different extreme - let us think of the $\tmu$ that is uniform over $(w,\ell)$ where $w$ is a uniformly random subset of $\alpha n$ of the $n$ samples and $\ell$ is a random linear function \footnote{The reader might be alarmed that such a linear function can never get the labels right. That is indeed what we want to show - that any distribution that satisfies the constraints of our program must be far from this construction.} correctly. In this case, it is easy to see that $v_i$ can be arbitrarily far from $\ell^*$. 

% In order for our guesses $v_i$ for $i \in \cI$ to be close to $\ell^*$, we must thus show that our distribution always ``looks like'' the first case rather than the second. We can phrase this more concretely by asking that whenever $(w,\ell)$ lies in the support of any distribution $\tmu$ that satisfies our constraints, then $\tmu$ must satisfy the following property: if $\|\ell -\ell^*\|_2 > \eta$, then, $w$ must only intersect $\cI$ in a small fraction of the points. 

% \paragraph{\eqref{eq:inliers-guess-well} from Littlewood-Offord Type Anti-Concentration} It turns out that the above statement is essentially a rephrasing of the distribution $D$ of the linear equations in $\cI$ satisfying a \emph{anti-concentration} inequality. To see why, consider a subset $A$ of size $\alpha n$ such that $C = |A \cap \cI|$ is non-empty. Suppose further that there's a linear function $\ell' \neq \ell^*$ that correctly labels samples in $A$. Then, $\ell- \ell^*$ is a non-zero vector that evaluates to $0$ on all of $C$. We would like to show that if $\ell$ is far from $\ell^*$ then $C$ cannot be large. More specifically, let $v = \ell - \ell^*$ and consider the points in $C$. Then, $C$ is a subset of a uniform sample from $D$ and $\iprod{x, v} = 0$ for all points in $C$. If the sample size $\alpha n$ is large enough, then, this must mean that $\Pr_{x \sim D} [\iprod{x,v} = 0 ]\geq \frac{|C|}{\alpha n}$. 

% Littlewood-Offord type inequalities show a tight (up to constants) bound on such probabilities. The simplest example is that of the standard gaussian distribution $\cN(0,I)$ where we have: $\Pr_{x \sim \cN(0,I)} [|\iprod{x,v}| < \delta]\leq O(\delta)$. Such a statement extends (via more non-trivial arguments) to other distributions such as the uniform distribution~\cite{ErdosLittlewoodOfford} on the hypercube $\on^n$. 

% Thus, if we could show that no non-zero vector evaluates to zero on a large enough fraction of points, we will immediately obtain any $A$ as above must be approximately disjoint from $\cI$. It turns out that this is indeed true in a robust way whenever the example points in the inliers are chosen from a distribution $D$ on $\R^d$ such that for any direction $v$, the random variable $\iprod{x,v}$ is sufficiently \emph{anti-concentrated} (so that the probability $\iprod{x,v} <\delta$ is small. 

 % Weaker quantitative versions of such a statement follow from variants of the classical Littlewood-Offord  anti-concentration inequalities that effectively say that for ``nice enough'' probability distributions $D$ on $\R^d$, and for any unit vector $v$, $\Pr[ |\iprod{ x, v}| < \delta \sqrt{\E \iprod{x,v}^2}] \leq O(\delta)$. 
 % Thus, it appears that at least for actual distributions $\tmu$, we can ensure \eqref{eq:good-weight-on-inliers} whenever $D$ satisfies an anti-concentration inequality.

\paragraph{SoSizing Anti-Concentration} The key to proving \eqref{eq:inliers-guess-well} for pseudo-distributions is a \emph{sum-of-squares} (SoS) proof of anti-concentration inequality: $\Pr_{x \sim \cI} [\iprod{x,v} =0] \leq \eta$ in variable $v$. SoS (see Section~\ref{sec:preliminaries}) is a restricted system for proving polynomial inequalities subject to polynomial inequality constraints. Thus, to even ask for a SoS proof we must phrase anti-concentration as a polynomial inequality. 

To do this, let $p(z)$ be a low-degree polynomial approximator for the function $\1\Paren{ z=0}$. 
Then, we can hope to ``replace'' the use of the inequality $\Pr_{x \sim \cI} [\iprod{x,v} =0] \leq \eta \equiv \E_{x \sim \cI} [\1(\iprod{x,v} = 0)] \leq \eta$ in the argument above by $\E_{x \sim \cI}[ p(\iprod{x,v})^2] \leq \eta$. Since  polynomials grow unboundedly for large enough inputs, it is \emph{necessary} for the uniform distribution on $\cI$ to have sufficiently light-tails to ensure that $\E_{x \sim \cI} p(\iprod{x,v})^2$ is small. In Lemma~\ref{lem:univppty_box}, we show that anti-concentration and light-tails are \emph{sufficient} to construct such a polynomial. 

We can finally ask for a SoS proof for $\E_{x \sim \cI} p(\iprod{x,v}) \leq \eta$ in variable $v$. We prove such \emph{certified} anti-concentration inequalities for broad families of inlier distributions in Section~\ref{sec:certified-anti-concentration}.

%USE FOR INTRO
% \paragraph{Comparison with the robust estimation framework} Our algorithm follows a line of recent works~\cite{KothariSteinhardt17,2017KS,HopkinsLi17,DBLP:conf/colt/KlivansKM18} that developed a general approach for robust estimation based on the sum-of-squares method. These algorithms work whenever there is a sum-of-squares proof for concentration formalized via appropriate upper bounds on the moments of the inliers. A key idea in our work is formalizing and proving \emph{anti-concentration} inequalities in the sum-of-squares framework. Prior works also provide a fairly general blueprint for the case when $\alpha \gg 1/2$. However, for the list-decodable setting, one of the key difficulties appears in ``rounding'' pseudo-distributions - that is, obtaining a list from the pseudo-distribution. We use the rounding via ``guesses'' to overcome these difficulties. 


% We show certified concentration inequalities for the gaussian distribution and extend it to some other standard distributions under additional restrictions. 

% hus, in addition to anti-concentration, we assume that the distribution $D$ has subgaussian tails in all directions\footnote{For  the standard gaussian distribution, one can show the related claim that for any $\beta$ and $y_1, y_2, \ldots, y_n$ and for a large enough $n$ chosen independently from such a distribution, any subset $S \subseteq [n]$ of size $\beta n$ satisfies $\frac{1}{|S|} \sum_{i \in S} \langle y_i, v \rangle^2 \geq \Omega(\beta \sqrt{\log(1/\beta)})$. A (certified) version of such a statement follows from our (certified) anti-concentration inequality. We omit the details in this version.}.

% Currently, we only know how to prove our certified anti-concentration inequality in all generality for all anti-concentrated (potentially non-product) gaussian measures on $\R^d$. If we restrict the set of direction $v$ to all vectors with coordinates in $\{0,1,-1\}$, then, we can prove a certified anti-concentration inequality for all anti-concentrated distributions on $\R^d$ that are, in addition, 1) product, 2) identically distributed in each coordinate and 3) have subgaussian tails. This allow us to solve random noisy linear equations on all distributions that satisfy the three conditions above under the additional promise that the unknown linear function has coordinates in $\{\frac{-1}{\sqrt{d}},\frac{1}{\sqrt{d}}\}$ (or tiny enough perturbations thereof).

% \paragraph{\eqref{eq:good-weight-on-inliers} from a Max-Entropy Constraint} Since we do not know the inliers, we cannot add this condition as a constraint. Instead, to force the algorithm to return a $\tmu$ with this property, we add a ``maximum entropy'' constraint instead. We describe the details of this below.The key observation that allows us to ensure \eqref{eq:good-weight-on-inliers} is that while an arbitrary $\tmu$ that satisfies our constraints need not ensure this property, a \emph{maximum entropy} $\tmu$ must! This is because for any (pseudo)-distribution that does not place enough ``weight'' on the inliers $\cI$, ``spreading'' some of the probability mass on to the inliers increases the entropy of the distribution. 


% % Intuitively speaking, if there are multiple subsets of the sample of size $\alpha n$ that are correctly labeled by a linear function, then we expect the maximum entropy (pseudo)-distribution to return the uniform distribution on all such subsets. Such a (pseudo)-distribution, thus, must have a non-zero probability mass on the inliers.

% Since the only meaningful information we have about pseudo-distributions are their low-degree moments, we must use a notion of ``entropy'' that is expressible in terms of just the low-degree moments of the pseudo-distribution. Our choice is extremely simple and uses just the first moment: we simply minimize the $\ell_2$ norm of $\pE[w]$. Note that this is a convex minimization problem and thus efficiently (approximately) soluble using the ellipsoid algorithm. 


%As one might guess, we will, in fact, need a robust version of such a statement that in fact shows that the maximum entropy pseudo-distributoin must place a significant ``weight'' on the true inliers. This is indeed true as we show when we analyze our algorithm. 



% In this work, we design algorithms for the problem of linear regression that are robust to training sets with an overwhelming ($\gg 1/2$) fraction of adversarially chosen outliers. 

% Outlier-robust learning algorithms have been extensively studied (under the name \emph{robust statistics}) in mathematical statistics~\cite{MR0426989,maronna2006robust,huber2011robust,hampel2011robust}. However, the algorithms resulting from this line of work usually run in time exponential in the dimension of the data~\cite{bernholt2006robust}. An influential line of recent work \cite{journals/jmlr/KlivansLS09,journals/corr/AwasthiBL13, DBLP:journals/corr/DiakonikolasKKL16,DBLP:conf/focs/LaiRV16,DBLP:conf/stoc/CharikarSV17,KothariSteinhardt17,2017KS,HopkinsLi17,DBLP:journals/corr/DiakonikolasKK017a,DBLP:journals/corr/DiakonikolasKS17,DBLP:conf/colt/KlivansKM18} has focused on designing \emph{efficient} algorithms for outlier-robust learning. 

% Our work extends this line of research. Our algorithms work in the \emph{list-decodable learning} framework. In this model, the majority of the training data (a $1 -\alpha$ fraction) can be adversarially corrupted leaving only an $\alpha \ll 1/2$ fraction of \emph{inliers}. Since uniquely recovering the underlying parameters is information-theoretically \emph{impossible} in such a setting, the goal is to output a list (with an absolute constant size) of parameters, one of which matches the ground truth. This model was introduced in~\cite{DBLP:conf/stoc/BalcanBV08} to give a discriminative framework for clustering. More recently, beginning with~\cite{DBLP:conf/stoc/CharikarSV17}, various works~\cite{DBLP:conf/stoc/DiakonikolasKS18,KothariSteinhardt17} have considered this as a model of \emph{untrusted} data. 

% There has been a phenomenal progress in developing techniques for outlier-robust learning with a \emph{small} $(\ll 1/2)$-fraction of outliers (e.g. outlier \emph{filters}~\cite{DBLP:conf/focs/DiakonikolasKK016,DBLP:journals/corr/DiakonikolasKK017a}, separation oracles for inliers~\cite{DBLP:conf/focs/DiakonikolasKK016} or the \emph{sum-of-squares} method~\cite{2017KS,HopkinsLi17,KothariSteinhardt17,DBLP:conf/colt/KlivansKM18}). In contrast, progress on algorithms that tolerate the significantly harsher conditions in the list-decodable setting has been slower. The only prior works~\cite{DBLP:conf/stoc/CharikarSV17,DBLP:conf/stoc/DiakonikolasKS18,KothariSteinhardt17} in this direction designed list-decodable algorithms for mean estimation via somewhat \emph{ad hoc}, problem-specific methods. 

% In this paper, we develop a principled technique to give the first efficient list-decodable learning algorithm for the fundamental problem of \emph{linear regression}. Our algorithm takes a corrupted set of linear equations with an $\alpha \ll 1/2$ fraction of inliers and outputs a $O(1/\alpha)$-size list of linear functions, one of which is guaranteed to be close to the ground truth (i.e., the linear function that correctly labels the inliers). Our key conceptual observation shows that list-decodable regression information-theoretically requires the inlier-distribution to be \emph{anti-concentrated}. Our algorithm succeeds whenever the distribution satisfies a stronger \emph{certifiable anti-concentration} condition. This class includes the standard gaussian distribution and more generally, any spherically symmetric distribution with strictly sub-exponential tails.

% Prior to our work\footnote{There's a long line of work on robust regression algorithms (see for e.g. \cite{DBLP:conf/nips/Bhatia0KK17,conf/soda/KarmalkarP19}) that can tolerate corruptions only in the \emph{labels}. We are interested in algorithms robust against corruptions in both examples and labels.}, the state-of-the-art outlier-robust algorithms for linear regression~\cite{DBLP:conf/colt/KlivansKM18,conf/soda/DiakonikolasKS19,journals/corr/abs-1803-02815,journals/corr/abs-1802-06485} could handle only a small $(<0.1)$-fraction of outliers even under strong assumptions on the underlying distributions. 

% List-decodable regression generalizes the well-studied~\cite{MR1028403,doi:10.1162/neco.1994.6.2.181,MR2757044,2013arXiv1310.3745Y,DBLP:journals/corr/BalakrishnanWY14,DBLP:conf/colt/ChenYC14,DBLP:conf/nips/Zhong0D16,DBLP:conf/aistats/SedghiJA16,DBLP:conf/colt/LiL18} and {\em easier} problem of \emph{mixed linear regression}: given $k$ ``clusters'' of examples that are labeled by one out of $k$ distinct unknown linear functions, find the unknown set of linear functions. All known techniques for the problem rely on faithfully estimating \emph{moment tensors} from samples and thus, cannot tolerate the overwhelming fraction of outliers in the list-decodable setting. On the other hand, since we can take any cluster as inliers and treat rest as outliers, our algorithm immediately yields new efficient algorithms for mixed linear regression. Unlike all prior works, our algorithms work without any pairwise separation or bounded condition-number assumptions on the $k$ linear functions. 








% % The MLR model is a popular mixture model and has many applications due to its effectiveness in captur- ing non-linearity and its model simplicity [De Veaux, 1989, Jordan and Jacobs, 1994, Faria and Soromenho, 2010, Zhong et al., 2016]. It has also been a recent theoretical topic for analyzing benchmark algorithms for nonconvex optimization (e.g., [Chaganty and Liang, 2013, Klusowski et al., 2017]) or designing new algo- rithms (e.g., [Chen et al., 2014]). However, most of the existing works either restrict to very special settings (e.g., x of different components all from the standard Gaussian, or only k = 2 components) [Chen et al., 2014, Yi et al., 2014, Zhong et al., 2016, Balakrishnan et al., 2017, Klusowski et al., 2017], or have high sample or computational complexity far from optimal [Chaganty and Liang, 2013, Sedghi et al., 2016].



% \paragraph{List-Decodable Learning via the Sum-of-Squares Method} Our algorithm relies on a strengthening of the robust-estimation framework based on the sum-of-squares (SoS) method. This paradigm has been recently used for clustering mixture models~\cite{HopkinsLi17,KothariSteinhardt17} and obtaining algorithms for moment estimation~\cite{2017KS} and linear regression~\cite{DBLP:conf/colt/KlivansKM18}
% relies on a strengthening of robust-estimation framework based on the sum-of-squares (SoS) method. This paradigm has been recently used for clustering mixture models~\cite{HopkinsLi17,KothariSteinhardt17} and obtaining algorithms for moment estimation~\cite{2017KS} and linear regression~\cite{DBLP:conf/colt/KlivansKM18} that are resilient to a small $(\ll 1/2)$ fraction of outliers under the mildest known assumptions on the underlying distributions. Ths method reduces outlier-robust algorithm design to finding ``simple'' proofs of  unique \emph{identifiability} of the unknown parameter of the original distribution from a corrupted sample. However, this principled method works only in the setting with a small ($\ll 1/2$) fraction of outliers. As a consequence, the work of~\cite{KothariSteinhardt17} for mean estimation in the list-decodable setting relied on ``supplementing'' the SoS method with a somewhat \emph{ad hoc}, problem-dependent technique. 

% As an important conceptual contribution, our work yields a framework for list-decodable learning that recovers some of the simplicity of the general blueprint. To do this, we give a general method for \emph{rounding pseudo-distributions} in the setting with $\gg 1/2$ fraction outliers. A key step in our rounding builds on the work of~\cite{KS19} who developed such a method to give a simpler proof of the list-decodable mean estimation result of~\cite{KothariSteinhardt17}. In Section~\ref{sec:overview}, we explain our ideas in detail. 

% The results in all the works above hold whenever the underlying distribution satisfies a certain \emph{certified concentration} condition formulated within the SoS system via higher moment bounds. An important contribution of this work is formalizing an \emph{anti-concentration} condition within the SoS system. Unlike the bounded moment condition, there is no canonical phrasing  within SoS for such statements. We choose a form that allows proving ``certified anti-concentration'' for a distribution by showing the existence of a certain approximating polynomial. This allows showing certified anti-concentration of natural distributions via a completely modular approach that relies on a beautiful line of works that construct ``weighted '' polynomial approximators~\cite{2007math......1099L}. 

% We believe that our framework for list-decodable estimation and our formulation of certified anti-concentration condition will likely have further applications in outlier-robust learning. 
% % We focus on the setting where the noise rate is larger than $1/2$ and unique recovery of the underlying parameters is information-theoretically {\em impossible}.  We study the supervised learning problem of linear regression and assume that a learner has been given a training set where a (possibly greater than $1/2$) fraction of examples have been {\em arbitrarily} corrupted in both the labels and locations of the data points.  Given this training set as input, we give the first efficient algorithm for outputting a {\em list} of linear functions, one of which agrees with the original, uncorrupted data set (the list is {\em constant-size} where the constant depends on the noise rate).  We refer to this problem as {\em list-decodable linear regression}\footnote{Our setting is similar in spirit to list-decodable error correcting codes, but the problem statements and techniques are quite different.}.  It is easy to see that list-decodable linear regression is harder than classical {\em mixed linear regression}, an intensely studied problem in machine learning.  Our algorithm requires an assumption on the marginal distribution, namely {\em certifiable anti-concentration}, which we prove is information-theoretically necessary for list-decodability in this setting.  
% \subsection{Our Results}
% We first define our model for generating samples for list-decodable regression.

% \begin{model}[Robust Linear Regression]
% 	For $0 <\alpha < 1$ and $\ell^* \in \R^d$ with $\|\ell^*\|_2 \leq 1$, let $\Lin_D(\alpha,\ell^*)$ denote the following probabilistic process to generate $n$ noisy linear equations $\cS = \{ \langle x_i, a \rangle = y_i\mid 1\leq i \leq n\}$ in variable $a \in \R^d$ with $\alpha n$ \emph{inliers} $\cI$ and $(1-\alpha)n$ \emph{outliers} $\cO$:
% 	\begin{enumerate}
% 		\item Construct $\cI$ by choosing $\alpha n$ i.i.d. samples $x_i \sim D$ and set $y_i = \langle x_i,\ell^* \rangle + \zeta$ for additive and independent noise $\zeta$,
% 		\item Construct $\cO$ by  choosing the remaining $(1-\alpha)n$ equations arbitrarily and potentially adversarially w.r.t the inliers $\cI$.
% 	\end{enumerate}
% 	\label{model:random-equations}
% \end{model}
% The bound on the norm of $\ell^*$ is without any loss of generality. 
% % Our algorithm naturally extends to handle additive noise in the labels. 
% % We will focus mostly on the noiseless case for clarity of exposition.
% Note that $\alpha$ is measure of the ``signal'' (fraction of inliers) and can be $\ll 1/2$. 

% An $\eta$-approximate algorithm for list-decodable regression takes input a sample from $\Lin_D(\alpha,\ell^*)$ and outputs a \emph{constant} (depending only on $\alpha$) size list $L$ of linear functions such that there is some $\ell \in L$ that is $\eta$-close to $\ell^*$.

% One of our key conceptual contributions is to identify the strong relationship between \emph{anti-concentration inequalities} and list-decodable regression. Anti-concentration inequalities are well-studied~\cite{ErdosLittlewoodOfford,MR2965282-Tao12,MR2407948-Rudelson08} in probability theory and combinatorics. The simplest of these inequalities upper bound the probability that a high-dimensional random variable has zero projections in any direction.

% \begin{definition}[Anti-Concentration]
% 	A $\R^d$-valued zero-mean random variable $Y$ has a $\delta$-\emph{anti-concentrated} distribution if $\Pr[ \iprod{Y,v}=0 ]< \delta$. 
% \end{definition}

% In Proposition~\ref{prop:identifiability}, we provide a simple but conceptually illuminating proof that anti-concentration is \emph{sufficient} for list-decodable regression. In Theorem~\ref{thm:main-lower-bound}, we prove a sharp converse and show that anti-concentration is information-theoretically \emph{necessary} for even noiseless list-decodable regression. This lower bound surprisingly holds for a natural distribution: uniform distribution on $\zo^d$ and more generally, uniform distribution on $[q]^d$ for  $q = \{0,1,2\ldots,q\}$.

% \begin{theorem}[See Proposition~\ref{prop:identifiability} and Theorem~\ref{thm:main-lower-bound}]
% 	There is a (inefficient) list-decodable regression algorithm for $\Lin_D(\alpha,\ell^*)$ with list size $O(\frac{1}{\alpha})$ whenever $D$ is $\alpha$-anti-concentrated. 
% 	Further, there exists a distribution $D$ on $\R^d$ that is $\paren{\alpha+\epsilon}$-anti-concentrated for every $\epsilon >0$ but there is no algorithm for $\frac{\alpha}{2}$-approximate list-decodable regression for $\Lin_D(\alpha,\ell^*)$  that returns a list of size $<d$. 
% \end{theorem}
% For our efficient algorithms, we need a \emph{certified} version of the anti-concentration condition. 
% To handle additive noise of variance $\zeta^2$, we need a control of $\Pr[ |\iprod{x,v}| \leq \zeta]$. 
% Thus, we extend our notion of anti-concentration and then define a \emph{certified} analog of it:
% \begin{definition}[Certifiable Anti-Concentration] \label{def:certified-anti-concentration}
% 	A random variable $Y$ has a $k$-\emph{certifiably} $(C,\delta)$-anti-concentrated distribution if there is a univariate polynomial $p$ satisfying $p(0) = 1$ such that there is a degree $k$ sum-of-squares proof of the following two inequalities: 
% 	\begin{enumerate} 
% 		\item $\forall v$, $\langle Y,v\rangle^2 \leq \delta^2 \E \langle Y,v\rangle^2$ implies $(p(\langle Y,v\rangle) -1)^2\leq \delta^2$.
% 		\item $\forall v$, $\|v\|_2^2 \leq 1$ implies  $\E p^2(\left \langle Y,v\rangle \right) \leq C\delta$.
% 	\end{enumerate}
% \end{definition} 
% Intuitively, certified anti-concentration asks for a \emph{certificate} of the anti-concentration property of $Y$ in the ``sum-of-squares'' proof system (see Section~\ref{sec:preliminaries} for precise definitions). SoS is a proof system that reasons about  polynomial inequalities. Since the ``core indicator'' $\1(|\iprod{x,v}| \leq \delta)$ is not a polynomial, we phrase the condition in terms of an approximating polynomial $p$.
% \Pnote{seems like it would be nice to have some explanation for the properties of the polynomial asked for in this definition...}
% We are now ready to state our main result.
% \begin{restatable}[List-Decodable Regression]{theorem}{main} \label{thm:main}
% 	For every $\alpha, \eta > 0$ and a $k$-certifiably $(C,\alpha^2 \eta^2/10C)$-anti-concentrated distribution $D$ on $\R^d$, there exists an algorithm that takes input a sample generated according to $\Lin_D(\alpha,\ell^*)$ and outputs a list $L$ of size $O(1/\alpha)$ such that there is an $\ell \in L$ satisfying $\| \ell - \ell^*\|_2 < \eta$ with probability at least $0.99$ over the draw of the sample. The algorithm needs a sample of size $n = (kd)^{O(k)}$ and runs in time $n^{O(k)} = (kd)^{O(k^2)}$.
% \end{restatable} 
% \begin{remark}[Tolerating Additive Noise]
% 	For additive noise of variance $\zeta^2$ in the inlier labels, our algorithm, in the same running time and sample complexity, outputs a list of size $O(1/\alpha)$ that contains an $\ell$ satisfying $\|\ell-\ell^*\|_2 \leq \frac{\zeta}{\alpha} + \eta$. Since we normalize $\ell^*$ to have unit norm, this guarantee is meaningful only when $\zeta \ll \alpha$. %It is not clear if better noise-tolerance is possible for list-decodable regression.
% \end{remark}

% \begin{remark}[Exponential Dependence on $1/\alpha$]
% 	List-decodable regression algorithms immediately yield algorithms for mixed linear regression (MLR) without any assumptions on the components. The state-of-the-art algorithm for MLR with gaussian components~\cite{DBLP:conf/colt/LiL18} has an exponential dependence on $k=1/\alpha$ in the running time in the absence of strong pairwise separation or small condition number of the components. Liang and Liu~\cite{DBLP:conf/colt/LiL18} (see Page 10) use the relationship to learning mixtures of $k$ gaussians (with an $\exp(k)$ lower bound~\cite{DBLP:conf/focs/MoitraV10}) to note that algorithms with polynomial dependence on $1/\alpha$ for MLR and thus, also for list-decodable regression might not exist. 
% \end{remark}


% \paragraph{Certifiably anti-concentrated distributions} In Section~\ref{sec:certified-anti-concentration}, we show certifiable anti-concentration of some well-studied families of distributions. This includes the standard gaussian distribution and more generally any anti-concentrated spherically symmetric distribution with strictly sub-exponential tails. We also show that simple operations such as scaling, applying well-conditioned linear transformations and sampling preserve certifiable anti-concentration. This yields:
% \begin{corollary}[List-Decodable Regression for Gaussian Inliers]
% 	For every $\alpha, \eta > 0$ there's an algorithm for list-decodable regression for the model $\Lin_D(\alpha,\ell^*)$ with $D = \cN(0,\Sigma)$ with $\lambda_{\max}(\Sigma)/\lambda_{min}(\Sigma) = O(1)$ that needs $n = (d/\alpha \eta)^{O\left(\frac{1}{\alpha^4 \eta^4}\right)}$  samples and runs in time $n^{O\left(\frac{1}{\alpha^4 \eta^4}\right)} = (d/\alpha \eta)^{O\left(\frac{1}{\alpha^8 \eta^8}\right)}$.
% \end{corollary} 

% We note that certifiably anti-concentrated distributions are more restrictive compared to the families of distributions for which the most general robust estimation algorithms work~\cite{2017KS,KothariSteinhardt17,DBLP:conf/colt/KlivansKM18}. To a certain extent, this is inherent. The families of distributions considered in these prior works do not satisfy anti-concentration in general.  And as we discuss in more detail in Section~\ref{sec:overview}, anti-concentration is information-theoretically \emph{necessary} (see Theorem~\ref{thm:main-lower-bound}) for list-decodable regression. This surprisingly rules out families of distributions that might appear natural and ``easy'', for example, the uniform distribution on $\zo^n$. In fact, our lower bound shows the impossibility of even the ``easier'' problem of mixed linear regression on this distribution.

% We rescue this to an extent for the special case when $\ell^*$ in the model $\Lin(\alpha,\ell^*)$ is a "Boolean vector", i.e., has all coordinates of equal magnitude. Intuitively, this helps because  while the the uniform distribution on $\zo^n$ (and more generally, any discrete product distribution) is badly anti-concentrated in sparse directions, they are well anti-concentrated~\cite{ErdosLittlewoodOfford} in the directions that are far from any sparse vectors. 

% As before, for obtaining efficient algorithms, we need to work with a \emph{certified} version (see Definition~\ref{def:certified-anti-concentration-Boolean}) of such a restricted anti-concentration condition. As a specific Corollary (see Theorem~\ref{thm:Booleanmain} for a more general statement), this allows us to show:
% \begin{theorem}[List-Decodable Regression for Hypercube Inliers] \label{thm:boolcube}
% 	For every $\alpha, \eta > 0$ there's an $\eta$-approximate algorithm for list-decodable regression for the model $\Lin_D(\alpha,\ell^*)$ with $D$ is uniform on $\zo^d$ that needs $n = (d/\alpha \eta)^{O(\frac{1}{\alpha^4 \eta^4})}$  samples and runs in time $n^{O(\frac{1}{\alpha^4 \eta^4})} = (d/\alpha \eta)^{O(\frac{1}{\alpha^8 \eta^8})}$.
% \end{theorem} 

% In Section~\ref{sec:hypercube}, we obtain similar results for general product distributions. It is an important open problem to prove certified anti-concentration for a broader family of distributions. % In particular, this yields a $d^{O(\frac{1}{\alpha^2 \eta^2})}$-sample and $d^{O(\frac{1}{\alpha^4 \eta^4})}$-time algorithm for list-decodable regression when $D$ is the uniform distribution on the Boolean hypercube/solid hypercube. 


% \subsection{Concurrent Work}
% In an independent and concurrent work, Raghavendra and Yau have given similar results for list-decodable linear regression and also use the sum-of-squares paradigm \cite{RY19}.

% \section{Preliminaries}
% \label{sec:preliminaries}

% In this section, we define pseudo-distributions and sum-of-squares proofs.
% See the lecture notes \cite{BarakS16} for more details and the appendix in \cite{DBLP:conf/focs/MaSS16} for proofs of the propositions appearing here.

% Let $x = (x_1, x_2, \ldots, x_n)$ be a tuple of $n$ indeterminates and let $\R[x]$ be the set of polynomials with real coefficients and indeterminates $x_1,\ldots,x_n$.
% We say that a polynomial $p\in \R[x]$ is a \emph{sum-of-squares (sos)} if there are polynomials $q_1,\ldots,q_r$ such that $p=q_1^2 + \cdots + q_r^2$.

% \subsection{Pseudo-distributions}

% Pseudo-distributions are generalizations of probability distributions.
% We can represent a discrete (i.e., finitely supported) probability distribution over $\R^n$ by its probability mass function $D\from \R^n \to \R$ such that $D \geq 0$ and $\sum_{x \in \mathrm{supp}(D)} D(x) = 1$.
% Similarly, we can describe a pseudo-distribution by its mass function.
% Here, we relax the constraint $D\ge 0$ and only require that $D$ passes certain low-degree non-negativity tests.

% Concretely, a \emph{level-$\ell$ pseudo-distribution} is a finitely-supported function $D:\R^n \rightarrow \R$ such that $\sum_{x} D(x) = 1$ and $\sum_{x} D(x) f(x)^2 \geq 0$ for every polynomial $f$ of degree at most $\ell/2$.
% (Here, the summations are over the support of $D$.)
% A straightforward polynomial-interpolation argument shows that every level-$\infty$-pseudo distribution satisfies $D\ge 0$ and is thus an actual probability distribution.
% We define the \emph{pseudo-expectation} of a function $f$ on $\R^d$ with respect to a pseudo-distribution $D$, denoted $\pE_{D(x)} f(x)$, as
% \begin{equation}
% \pE_{D(x)} f(x) = \sum_{x} D(x) f(x) \,\mper
% \end{equation}
% The degree-$\ell$ moment tensor of a pseudo-distribution $D$ is the tensor $\E_{D(x)} (1,x_1, x_2,\ldots, x_n)^{\otimes \ell}$.
% In particular, the moment tensor has an entry corresponding to the pseudo-expectation of all monomials of degree at most $\ell$ in $x$.
% The set of all degree-$\ell$ moment tensors of probability distribution is a convex set.
% Similarly, the set of all degree-$\ell$ moment tensors of degree $d$ pseudo-distributions is also convex.
% Key to the algorithmic utility of pseudo-distributions is the fact that while there can be no efficient separation oracle for the convex set of all degree-$\ell$ moment tensors of an actual probability distribution, there's a separation oracle running in time $n^{O(\ell)}$ for the convex set of the degree-$\ell$ moment tensors of all level-$\ell$ pseudodistributions.

% \begin{fact}[\cite{MR939596-Shor87,parrilo2000structured,MR1748764-Nesterov00,MR1846160-Lasserre01}]
% 	\label[fact]{fact:sos-separation-efficient}
% 	For any $n,\ell \in \N$, the following set has a $n^{O(\ell)}$-time weak separation oracle (in the sense of \cite{MR625550-Grotschel81}):
% 	\begin{equation}
% 	\Set{ \pE_{D(x)} (1,x_1, x_2, \ldots, x_n)^{\otimes d} \mid \text{ degree-d pseudo-distribution $D$ over $\R^n$}}\,\mper
% 	\end{equation}
% \end{fact}
% This fact, together with the equivalence of weak separation and optimization \cite{MR625550-Grotschel81} allows us to efficiently optimize over pseudo-distributions (approximately)---this algorithm is referred to as the sum-of-squares algorithm.

% The \emph{level-$\ell$ sum-of-squares algorithm} optimizes over the space of all level-$\ell$ pseudo-distributions that satisfy a given set of polynomial constraints---we formally define this next.

% \begin{definition}[Constrained pseudo-distributions]
% 	Let $D$ be a level-$\ell$ pseudo-distribution over $\R^n$.
% 	Let $\cA = \{f_1\ge 0, f_2\ge 0, \ldots, f_m\ge 0\}$ be a system of $m$ polynomial inequality constraints.
% 	We say that \emph{$D$ satisfies the system of constraints $\cA$ at degree $r$}, denoted $D \sdtstile{r}{} \cA$, if for every $S\subseteq[m]$ and every sum-of-squares polynomial $h$ with $\deg h + \sum_{i\in S} \max\set{\deg f_i,r}$,
% 	\begin{displaymath}
% 	\pE_{D} h \cdot \prod _{i\in S}f_i  \ge 0\,.
% 	\end{displaymath}
% 	We write $D \sdtstile{}{} \cA$ (without specifying the degree) if $D \sdtstile{0}{} \cA$ holds.
% 	Furthermore, we say that $D\sdtstile{r}{}\cA$ holds \emph{approximately} if the above inequalities are satisfied up to an error of $2^{-n^\ell}\cdot \norm{h}\cdot\prod_{i\in S}\norm{f_i}$, where $\norm{\cdot}$ denotes the Euclidean norm\footnote{The choice of norm is not important here because the factor $2^{-n^\ell}$ swamps the effects of choosing another norm.} of the cofficients of a polynomial in the monomial basis.
% \end{definition}

% We remark that if $D$ is an actual (discrete) probability distribution, then we have  $D\sdtstile{}{}\cA$ if and only if $D$ is supported on solutions to the constraints $\cA$.

% We say that a system $\cA$ of polynomial constraints is \emph{explicitly bounded} if it contains a constraint of the form $\{ \|x\|^2 \leq M\}$.
% The following fact is a consequence of \cref{fact:sos-separation-efficient} and \cite{MR625550-Grotschel81},

% \begin{fact}[Efficient Optimization over Pseudo-distributions]
% 	There exists an $(n+ m)^{O(\ell)} $-time algorithm that, given any explicitly bounded and satisfiable system\footnote{Here, we assume that the bitcomplexity of the constraints in $\cA$ is $(n+m)^{O(1)}$.} $\cA$ of $m$ polynomial constraints in $n$ variables, outputs a level-$\ell$ pseudo-distribution that satisfies $\cA$ approximately. \label{fact:eff-pseudo-distribution}
% \end{fact}

% \subsection{Sum-of-squares proofs}

% Let $f_1, f_2, \ldots, f_r$ and $g$ be multivariate polynomials in $x$.
% A \emph{sum-of-squares proof} that the constraints $\{f_1 \geq 0, \ldots, f_m \geq 0\}$ imply the constraint $\{g \geq 0\}$ consists of  polynomials $(p_S)_{S \subseteq [m]}$ such that
% \begin{equation}
% g = \sum_{S \subseteq [m]} p_S \cdot \Pi_{i \in S} f_i
% \mper
% \end{equation}
% We say that this proof has \emph{degree $\ell$} if for every set $S \subseteq [m]$, the polynomial $p_S \Pi_{i \in S} f_i$ has degree at most $\ell$.
% If there is a degree $\ell$ SoS proof that $\{f_i \geq 0 \mid i \leq r\}$ implies $\{g \geq 0\}$, we write:
% \begin{equation}
% \{f_i \geq 0 \mid i \leq r\} \sststile{\ell}{}\{g \geq 0\}
% \mper
% \end{equation}


% Sum-of-squares proofs satisfy the following inference rules.
% For all polynomials $f,g\colon\R^n \to \R$ and for all functions $F\colon \R^n \to \R^m$, $G\colon \R^n \to \R^k$, $H\colon \R^{p} \to \R^n$ such that each of the coordinates of the outputs are polynomials of the inputs, we have:

% \begin{align}
% &\frac{\cA \sststile{\ell}{} \{f \geq 0, g \geq 0 \} } {\cA \sststile{\ell}{} \{f + g \geq 0\}}, \frac{\cA \sststile{\ell}{} \{f \geq 0\}, \cA \sststile{\ell'}{} \{g \geq 0\}} {\cA \sststile{\ell+\ell'}{} \{f \cdot g \geq 0\}} \tag{addition and multiplication}\\
% &\frac{\cA \sststile{\ell}{} \cB, \cB \sststile{\ell'}{} C}{\cA \sststile{\ell \cdot \ell'}{} C}  \tag{transitivity}\\
% &\frac{\{F \geq 0\} \sststile{\ell}{} \{G \geq 0\}}{\{F(H) \geq 0\} \sststile{\ell \cdot \deg(H)} {} \{G(H) \geq 0\}} \tag{substitution}\mper
% \end{align}

% Low-degree sum-of-squares proofs are sound and complete if we take low-level pseudo-distributions as models.

% Concretely, sum-of-squares proofs allow us to deduce properties of pseudo-distributions that satisfy some constraints.

% \begin{fact}[Soundness]
% 	\label{fact:sos-soundness}
% 	If $D \sdtstile{r}{} \cA$ for a level-$\ell$ pseudo-distribution $D$ and there exists a sum-of-squares proof $\cA \sststile{r'}{} \cB$, then $D \sdtstile{r\cdot r'+r'}{} \cB$.
% \end{fact}

% If the pseudo-distribution $D$ satisfies $\cA$ only approximately, soundness continues to hold if we require an upper bound on the bit-complexity of the sum-of-squares $\cA \sststile{r'}{} B$  (number of bits required to write down the proof).

% In our applications, the bit complexity of all sum of squares proofs will be $n^{O(\ell)}$ (assuming that all numbers in the input have bit complexity $n^{O(1)}$).
% This bound suffices in order to argue about pseudo-distributions that satisfy polynomial constraints approximately.

% The following fact shows that every property of low-level pseudo-distributions can be derived by low-degree sum-of-squares proofs.

% \begin{fact}[Completeness]
% 	\label{fact:sos-completeness}
% 	Suppose $d \geq r' \geq r$ and $\cA$ is a collection of polynomial constraints with degree at most $r$, and $\cA \vdash \{ \sum_{i = 1}^n x_i^2 \leq B\}$ for some finite $B$.
	
% 	Let $\{g \geq 0 \}$ be a polynomial constraint.
% 	If every degree-$d$ pseudo-distribution that satisfies $D \sdtstile{r}{} \cA$ also satisfies $D \sdtstile{r'}{} \{g \geq 0 \}$, then for every $\epsilon > 0$, there is a sum-of-squares proof $\cA \sststile{d}{} \{g \geq - \epsilon \}$.
% \end{fact}

% We will use the following Cauchy-Schwarz inequality for pseudo-distributions:

% \begin{fact}[Cauchy-Schwarz for Pseudo-distributions]
% 	Let $f,g$ be polynomials of degree at most $d$ in indeterminate $x \in \R^d$. Then, for any degree d pseudo-distribution $\tmu$,
% 	$\pE_{\tmu}[fg] \leq \sqrt{\pE_{\tmu}[f^2]} \sqrt{\pE_{\tmu}[g^2]}$.
% 	\label{fact:pseudo-expectation-cauchy-schwarz}
% \end{fact} 

% The following fact is a simple corollary of the fundamental theorem of algebra:
% \begin{fact}
% 	For any univariate degree $d$ polynomial $p(x) \geq 0$ for all $x \in \R$, 
% 	$\sststile{d}{x} \Set{p(x) \geq 0}$.
% 	\label{fact:univariate}
% \end{fact}

% This can be extended to univariate polynomial inequalities over intervals of $\R$. 

% \begin{fact}[Fekete and Markov-Lukács, see \cite{laurent2009sums}]
% 	For any univariate degree $d$ polynomial $p(x) \geq 0$ for $x \in [a, b]$,  $\Set{x\geq a, x \leq b} \sststile{d}{x} \Set{p(x) \geq 0}$.  \label{fact:univariate-interval}
% \end{fact}



\phantomsection
\addcontentsline{toc}{section}{References}
\bibliographystyle{plain}
\bibliography{bib/mathreview,bib/dblp,bib/custom,bib/scholar,bib/main}

\end{document}