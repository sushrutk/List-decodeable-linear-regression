%!TEX root = ../main.tex

\section{Preliminaries}
\label{sec:preliminaries}
% The \emph{symmetrized Hausdorff distance} between two sets $S_0,S_1 \subseteq \R^n$ is defined as the $ \max_{b \in \{0,1\}} \min_{s \in S_b} \max_{t \in S_{1-b}} \frac{\langle s,t \rangle^2}{\|s\|^2\|t\|^2} \geq 1-\epsilon$. We say that two matrices $A, A'$ are $\epsilon$-close in symmetrized Hausdorff distance if the set of columns of $A, A'$ are $\epsilon$-close in symmetrized Hausdorff distance.


% We write $\Perm{[n]}{r}$ for the set of all $r$-tuples formed by taking $r$ \emph{distinct} elements from $[n]$. The Kronecker product $T$ of $v_1 \otimes v_2 \otimes \ldots v_k$ for vectors $v_i \in \R^{n_i}$ is a vector in $\R^{n_1 \times n_2 \times \cdots n_k}$ with entries indexed by tuples $(i_1, i_2, \ldots, i_k)$ where any $i_j \in [n_j]$ defined by $T(i_1, i_2, \ldots, i_k) = \Pi_{j \leq k} v_j(i_j)$. $\|\cdot \|$ for matrices will denote the spectral norm and for vectors, the Euclidean norm.

In this section, we define pseudo-distributions and sum-of-squares proofs.
See the lecture notes \cite{BarakS16} for more details and the appendix in \cite{DBLP:conf/focs/MaSS16} for proofs of the propositions appearing here.

Let $x = (x_1, x_2, \ldots, x_n)$ be a tuple of $n$ indeterminates and let $\R[x]$ be the set of polynomials with real coefficients and indeterminates $x_1,\ldots,x_n$.
We say that a polynomial $p\in \R[x]$ is a \emph{sum-of-squares (sos)} if there are polynomials $q_1,\ldots,q_r$ such that $p=q_1^2 + \cdots + q_r^2$.

\subsection{Pseudo-distributions}

Pseudo-distributions are generalizations of probability distributions.
We can represent a discrete (i.e., finitely supported) probability distribution over $\R^n$ by its probability mass function $D\from \R^n \to \R$ such that $D \geq 0$ and $\sum_{x \in \mathrm{supp}(D)} D(x) = 1$.
Similarly, we can describe a pseudo-distribution by its mass function.
Here, we relax the constraint $D\ge 0$ and only require that $D$ passes certain low-degree non-negativity tests.

Concretely, a \emph{level-$\ell$ pseudo-distribution} is a finitely-supported function $D:\R^n \rightarrow \R$ such that $\sum_{x} D(x) = 1$ and $\sum_{x} D(x) f(x)^2 \geq 0$ for every polynomial $f$ of degree at most $\ell/2$.
(Here, the summations are over the support of $D$.)
A straightforward polynomial-interpolation argument shows that every level-$\infty$-pseudo distribution satisfies $D\ge 0$ and is thus an actual probability distribution.
We define the \emph{pseudo-expectation} of a function $f$ on $\R^d$ with respect to a pseudo-distribution $D$, denoted $\pE_{D(x)} f(x)$, as
\begin{equation}
  \pE_{D(x)} f(x) = \sum_{x} D(x) f(x) \,\mper
\end{equation}
The degree-$\ell$ moment tensor of a pseudo-distribution $D$ is the tensor $\E_{D(x)} (1,x_1, x_2,\ldots, x_n)^{\otimes \ell}$.
In particular, the moment tensor has an entry corresponding to the pseudo-expectation of all monomials of degree at most $\ell$ in $x$.
The set of all degree-$\ell$ moment tensors of probability distribution is a convex set.
Similarly, the set of all degree-$\ell$ moment tensors of degree $d$ pseudo-distributions is also convex.
Key to the algorithmic utility of pseudo-distributions is the fact that while there can be no efficient separation oracle for the convex set of all degree-$\ell$ moment tensors of an actual probability distribution, there's a separation oracle running in time $n^{O(\ell)}$ for the convex set of the degree-$\ell$ moment tensors of all level-$\ell$ pseudodistributions.

\begin{fact}[\cite{MR939596-Shor87,parrilo2000structured,MR1748764-Nesterov00,MR1846160-Lasserre01}]
  \label[fact]{fact:sos-separation-efficient}
  For any $n,\ell \in \N$, the following set has a $n^{O(\ell)}$-time weak separation oracle (in the sense of \cite{MR625550-Grotschel81}):
  \begin{equation}
    \Set{ \pE_{D(x)} (1,x_1, x_2, \ldots, x_n)^{\otimes d} \mid \text{ degree-d pseudo-distribution $D$ over $\R^n$}}\,\mper
  \end{equation}
\end{fact}
This fact, together with the equivalence of weak separation and optimization \cite{MR625550-Grotschel81} allows us to efficiently optimize over pseudo-distributions (approximately)---this algorithm is referred to as the sum-of-squares algorithm.

The \emph{level-$\ell$ sum-of-squares algorithm} optimizes over the space of all level-$\ell$ pseudo-distributions that satisfy a given set of polynomial constraints---we formally define this next.

\begin{definition}[Constrained pseudo-distributions]
  Let $D$ be a level-$\ell$ pseudo-distribution over $\R^n$.
  Let $\cA = \{f_1\ge 0, f_2\ge 0, \ldots, f_m\ge 0\}$ be a system of $m$ polynomial inequality constraints.
  We say that \emph{$D$ satisfies the system of constraints $\cA$ at degree $r$}, denoted $D \sdtstile{r}{} \cA$, if for every $S\subseteq[m]$ and every sum-of-squares polynomial $h$ with $\deg h + \sum_{i\in S} \max\set{\deg f_i,r}$,
  \begin{displaymath}
    \pE_{D} h \cdot \prod _{i\in S}f_i  \ge 0\,.
  \end{displaymath}
  We write $D \sdtstile{}{} \cA$ (without specifying the degree) if $D \sdtstile{0}{} \cA$ holds.
  Furthermore, we say that $D\sdtstile{r}{}\cA$ holds \emph{approximately} if the above inequalities are satisfied up to an error of $2^{-n^\ell}\cdot \norm{h}\cdot\prod_{i\in S}\norm{f_i}$, where $\norm{\cdot}$ denotes the Euclidean norm\footnote{The choice of norm is not important here because the factor $2^{-n^\ell}$ swamps the effects of choosing another norm.} of the cofficients of a polynomial in the monomial basis.
\end{definition}

We remark that if $D$ is an actual (discrete) probability distribution, then we have  $D\sdtstile{}{}\cA$ if and only if $D$ is supported on solutions to the constraints $\cA$.

We say that a system $\cA$ of polynomial constraints is \emph{explicitly bounded} if it contains a constraint of the form $\{ \|x\|^2 \leq M\}$.
The following fact is a consequence of \cref{fact:sos-separation-efficient} and \cite{MR625550-Grotschel81},

\begin{fact}[Efficient Optimization over Pseudo-distributions]
There exists an $(n+ m)^{O(\ell)} $-time algorithm that, given any explicitly bounded and satisfiable system\footnote{Here, we assume that the bitcomplexity of the constraints in $\cA$ is $(n+m)^{O(1)}$.} $\cA$ of $m$ polynomial constraints in $n$ variables, outputs a level-$\ell$ pseudo-distribution that satisfies $\cA$ approximately. \label{fact:eff-pseudo-distribution}
\end{fact}

\subsection{Sum-of-squares proofs}

Let $f_1, f_2, \ldots, f_r$ and $g$ be multivariate polynomials in $x$.
A \emph{sum-of-squares proof} that the constraints $\{f_1 \geq 0, \ldots, f_m \geq 0\}$ imply the constraint $\{g \geq 0\}$ consists of  polynomials $(p_S)_{S \subseteq [m]}$ such that
\begin{equation}
g = \sum_{S \subseteq [m]} p_S \cdot \Pi_{i \in S} f_i
\mper
\end{equation}
We say that this proof has \emph{degree $\ell$} if for every set $S \subseteq [m]$, the polynomial $p_S \Pi_{i \in S} f_i$ has degree at most $\ell$.
If there is a degree $\ell$ SoS proof that $\{f_i \geq 0 \mid i \leq r\}$ implies $\{g \geq 0\}$, we write:
\begin{equation}
  \{f_i \geq 0 \mid i \leq r\} \sststile{\ell}{}\{g \geq 0\}
  \mper
\end{equation}


Sum-of-squares proofs satisfy the following inference rules.
For all polynomials $f,g\colon\R^n \to \R$ and for all functions $F\colon \R^n \to \R^m$, $G\colon \R^n \to \R^k$, $H\colon \R^{p} \to \R^n$ such that each of the coordinates of the outputs are polynomials of the inputs, we have:

\begin{align}
&\frac{\cA \sststile{\ell}{} \{f \geq 0, g \geq 0 \} } {\cA \sststile{\ell}{} \{f + g \geq 0\}}, \frac{\cA \sststile{\ell}{} \{f \geq 0\}, \cA \sststile{\ell'}{} \{g \geq 0\}} {\cA \sststile{\ell+\ell'}{} \{f \cdot g \geq 0\}} \tag{addition and multiplication}\\
&\frac{\cA \sststile{\ell}{} \cB, \cB \sststile{\ell'}{} C}{\cA \sststile{\ell \cdot \ell'}{} C}  \tag{transitivity}\\
&\frac{\{F \geq 0\} \sststile{\ell}{} \{G \geq 0\}}{\{F(H) \geq 0\} \sststile{\ell \cdot \deg(H)} {} \{G(H) \geq 0\}} \tag{substitution}\mper
\end{align}

Low-degree sum-of-squares proofs are sound and complete if we take low-level pseudo-distributions as models.

Concretely, sum-of-squares proofs allow us to deduce properties of pseudo-distributions that satisfy some constraints.

\begin{fact}[Soundness]
  \label{fact:sos-soundness}
  If $D \sdtstile{r}{} \cA$ for a level-$\ell$ pseudo-distribution $D$ and there exists a sum-of-squares proof $\cA \sststile{r'}{} \cB$, then $D \sdtstile{r\cdot r'+r'}{} \cB$.
\end{fact}

If the pseudo-distribution $D$ satisfies $\cA$ only approximately, soundness continues to hold if we require an upper bound on the bit-complexity of the sum-of-squares $\cA \sststile{r'}{} B$  (number of bits required to write down the proof).

In our applications, the bit complexity of all sum of squares proofs will be $n^{O(\ell)}$ (assuming that all numbers in the input have bit complexity $n^{O(1)}$).
This bound suffices in order to argue about pseudo-distributions that satisfy polynomial constraints approximately.

The following fact shows that every property of low-level pseudo-distributions can be derived by low-degree sum-of-squares proofs.

\begin{fact}[Completeness]
  \label{fact:sos-completeness}
  Suppose $d \geq r' \geq r$ and $\cA$ is a collection of polynomial constraints with degree at most $r$, and $\cA \vdash \{ \sum_{i = 1}^n x_i^2 \leq B\}$ for some finite $B$.

  Let $\{g \geq 0 \}$ be a polynomial constraint.
  If every degree-$d$ pseudo-distribution that satisfies $D \sdtstile{r}{} \cA$ also satisfies $D \sdtstile{r'}{} \{g \geq 0 \}$, then for every $\epsilon > 0$, there is a sum-of-squares proof $\cA \sststile{d}{} \{g \geq - \epsilon \}$.
\end{fact}

We will use the following Cauchy-Schwarz inequality for pseudo-distributions:

\begin{fact}[Cauchy-Schwarz for Pseudo-distributions]
Let $f,g$ be polynomials of degree at most $d$ in indeterminate $x \in \R^d$. Then, for any degree d pseudo-distribution $\tmu$,
$\pE_{\tmu}[fg] \leq \sqrt{\pE_{\tmu}[f^2]} \sqrt{\pE_{\tmu}[g^2]}$.
 \label{fact:pseudo-expectation-cauchy-schwarz}
\end{fact} 

The following fact is a simple corollary of the fundamental theorem of algebra:
\begin{fact}
For any univariate degree $d$ polynomial $p(x) \geq 0$ for all $x \in \R$, 
$\sststile{d}{x} \Set{p(x) \geq 0}$.
 \label{fact:univariate}
\end{fact}

This can be extended to univariate polynomial inequalities over intervals of $\R$. 

\begin{fact}[Fekete and Markov-Lukács, see \cite{laurent2009sums}]
For any univariate degree $d$ polynomial $p(x) \geq 0$ for $x \in [a, b]$,  $\Set{x\geq a, x \leq b} \sststile{d}{x} \Set{p(x) \geq 0}$.  \label{fact:univariate-interval}
\end{fact}


% \begin{fact}[Basis SoS Proofs]  \label{fact:sos-hypercontractivity}
% Let $\mu$ be a distribution on $\R^n$ such that: 
% \begin{enumerate}
% \item $\E[x(i)^\tau] = 1$ for every even $\tau \leq d$.
% \item $\E[ x^{\alpha}] =0$ whenever $\alpha$ has some odd individual degree, $|\alpha|\leq d$.
% \item $|\E[ x^{\alpha}]| \leq \beta$ for every $\alpha$ that is even, $|\alpha| \leq d$.
% \end{enumerate}
% Then, $\E[\langle x, u \rangle^{2\tau}] \leq \beta \cdot \tau^{2\tau} \E[\langle x, u \rangle^2] ^{\tau}$ for any $\tau \leq d/2$.
% \end{fact}




% \paragraph{Notation}
% For polynomial $p, q$ of degree at most $d$ in $v = (v_1, v_2, \ldots, v_n)$, we say that $p \preceq q$ if $q(v)-p(v)$ is a sum-of-squares polynomials in $v$.

% \subsection{Sum-of-Squares Norms}
% The convex set of moment tensors of degree $d$ pseudo-distributions naturally defines a norm on $d$-tensors over $\R^{n}$. 

% \begin{definition}[$sos_k$-norm] \label{def:SoS-norm}
% Given a tensor $T \in (\R^{n})^{\otimes d}$, we define the $sos_k$ ($k \geq d$) norm of $T$, denoted by $\|T\|_{\sos_k}$  and defined by
% \[
% \|T\|^2_{\sos_k} = \sup_{D: D \vDash_k \{\|u\|^2  = 1\}} \pE_{D(u)} [T(u)^2].
% \]
% \end{definition}

% It is instructive to compare the $\sos_k$ norm with the injective tensor norm of $T$:

% \begin{definition}[Symmetric Injective Tensor Norm]
% Let $T \in (\R^{n})^{\otimes d}$ be a $d$-dimension tensor on $\R^n$. The injective tensor norm of $T$ is defined as 
% \[
% \|T\|^2_{inj} = \sup_{x \in \bbS^{n-1}} T(x)^2.
% \]
% That is, $\|T\|^2_{inj}$ is the supremum of the square of the homogenous degree $d$ polynomial whose coefficients are the entries of $T$.  \footnote{Without the symmetry constraints, the injective tensor norm of $T$ is defined as the supremum over the polynomial $\langle T, \otimes_{i \leq d} x_i \rangle$ where $x_i$ are $n$ dimensional unit vectors which is also a well-studied notion. We omit further discussion here but direct the reader to \cite{}.}
% \end{definition}
% The (symmetric) injective tensor norm of $T$ is NP-hard to compute in general and can be thought of as the ``optimization'' version of the problem of constructing a separation oracle for the convex set of all moment tensors of an actual probability distribution. 

% It is simple to verify that the above definition is equivalent to saying 
% \[
% \|T\|^2_{inj} = \sup_{D: D\vDash_{\infty} \{\|x\|^2 \geq 0\}}  \E_{D}[ T(x)^2] 
% \]

% Thus, the $\sos_k$ norm of $T$ is a relaxation of the injective tensor and can be efficiently computed following the discussion from the previous section.



% \subsection{Concentration Results}
% We will need some standard concentration results. 

% The following gives concentration bounds in spectral norm for sums of independent rank 1 matrices.
% \begin{fact}[See Vershynin] \label{fact:vershynin-conc}
% Let $z_1, z_2, \ldots, z_m$ be independent random variables in $\R^n$ with identical covariance $\E[z_iz_i^{\top}] = \Sigma$. Let $\|z_i\| \leq \sqrt{R}$ almost surely for all $i$. Then, there's a constant $c > 0$ such that for every $t \geq 0$, with probability at least $1-n \exp(-ct^2),$
% \[
% \| \frac{1}{m} \sum_{i = 1}^m z_i z_i^{\top} - \Sigma  \| \leq \max (\|\Sigma\|^{1/2} \gamma, \gamma^2),
% \]
% where $\gamma = t \sqrt{R/m}$.
% \end{fact}

% The following stronger concentration result holds when individual rank 1 terms are from a subgaussian distribution. 
% \begin{fact}[\cite{}] \label{fact:subgaussian-conc}
% Let $x \in \R^n$ have a subgaussian distribution with constant subgaussian norm and covariance $\Sigma$. Let $x_1, x_2, \ldots, x_m$ be independent samples from $x$.

% Then, with probability at least $1- 2e^{-nt^2}$ and $m = \Omega( (t/\epsilon)^2 n)$, $\| \frac{1}{m} \sum_{i = 1}^m x_i x_i^{\top}  - \Sigma\| \leq \epsilon$. 
% \end{fact}

% We will need the following concentration results for polynomials on the gaussian distribution and uniform distribution on $\on^n$. 

% \begin{fact}[Hypercontractivity\cite{}] \label{fact:hypercontractivity}
% Let $f(z))$ be a degree $d$ polynomial. Then,
% $$\Pr_{z}[ |f(z) - \E[f(z)]| \geq t] \leq e^2 \cdot \exp(- (\frac{t^2}{RVar[f]})^{1/d}),$$ where $Var[f] = \E[f^2]$ and $R$ is some absolute constant. Here, $z$ is a random vector with each coordinate $z(i)$ being distributed independently as $\cN(0,1)$ or uniform over $\on$.
% \end{fact}

% % \begin{fact}[Concentration of Sum of Independent Subgaussian Rank 1 Terms] \label{fact:subgaussian-conc}

% % Let $z_1,z_2, \ldots, z_m$ be independent random vectors in $\R^n$ satisfying: $\E[ z_i z_i^{\top}] = 0$ and $\E[ exp(u^{\top} x_i )] \leq e^{\|u\|_2^2 \gamma/2}$ for every $u \in \R^n$ for every $1 \leq i \leq n$ almost surely. 
% % % For all $\epsilon_0 \in (0,1/2)$ and $\delta \in (0,1)$,
% % % $\Pr[ \| \frac{1}{m} \sum_{i = 1}^m z_i z_i^{\top} \| > \frac{1}{1-2\epsilon_0} \zeta] \leq \delta$ where $$\zeta = \gamma \cdot \left(     \sqrt{\frac{32 n \log{(1+2/\epsilon_0)} + 32 \log{(2/\delta)}}{m}} + \frac{2 n\log{(1+2/\epsilon_0)} + \log{(2/\delta)} }{m} \right)$$.
% % For every $\delta \in (0,1)$,
% % % In particular, for $\epsilon_0  = \frac{1}{4}$, the above bound says:
% % $\Pr[ \| \frac{1}{m} \sum_{i = 1}^m z_i z_i^{\top} \| > 2 \zeta] \leq \delta$ where $$\zeta \leq 10\gamma \cdot \left(   \sqrt{ \frac{n + \log{(2/\delta)}}{m}} + \frac{2 n + \log{(2/\delta)} }{m} \right)$.$

% % \end{fact}

